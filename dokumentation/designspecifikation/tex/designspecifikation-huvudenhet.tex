\setcounter{secnumdepth}{5}
\section{Huvudmodul}
Huvudmodulen utför de flesta beräkningar nödvändiga för att roboten ska kunna utföra sina uppgifter. Dessa uppgifter ska huvudmodulen hantera antingen via instruktioner från PC-enheten eller helt autonomt. Detta är en kritisk modul då den kommer att utföra många uppgifter. Den behöver inte mycket hårdvara men den kommer innehålla majoriteten av robotens programvara.

\subsection{Hårdvara}
Modulen ska bestå av en enkortsdator av modell Beagleboard-xM. Den har en ARM Cortex-A8 processor som har en klockfrekvens på 1GHz. Denna behöver ett operativsystem för att kunna användas. Den enda hårdvaran som behövs för att kunna använda Beagleboard är ett minneskort för operativsystem och en Blåtandsdongel för kommunikationen med PC-enheten. Kommunikation med sensorenhet och styrenhet sker över SPI.

\subsection{Mjukvara}
Mjukvaran som behövs för att implementera all funktionalitet hos huvudmodulen kommer att vara skriven i programspråket Python. Koden för programmet kommer vara trådbaserad och möjligen objektorienterad.
\newline
Programmet ska delas in i 3 trådar. Dessa körs kontinuerligt och delar på två trådsäkra listor: sensorvärden och instruktioner. I instruktionslistanlistan ligger alla argument till tillhörande instruktioner från PC-enheten och i sensorvärden ligger all data från sensorenheten. Listorna kan trådarna antingen läsa eller skriva till (se figur \ref{designspec:huvudmodul-tradar}).

\begin{figure}[h]
\scalebox{0.8}{\input{grafik/huvud-tradar}}
\caption{Trådarna och listorna de delar på} \label{designspec:huvudmodul-tradar}
\end{figure}

Det som behövs göras på Beagleboard innan den kan börja användas är:
\begin{itemize}
\item Installera Angstrom operativsystem
\item Uppdatera operativsystem
\item Installera Python
\item Konfigurera SPI
\item Patcha kärnan för att stödja SPI
\item Installera Git
\item Installera utvecklingsmiljö
\item Installera Blåtandsmodulen
\item Sätta upp ett PAN
\end{itemize}

\subsubsection{Huvudtråden}
I huvudtråden ligger huvudloopen för programmet. Den börjar med att läsa värdet från instruktionslistan där manuellt eller autonomt läge avgörs och anropar respektive funktion. Användaren ska alltså kunna växla mellan dessa lägen när denne vill (mellan varje iteration av huvudloopen). Figur \ref{designspec:huvudmodul-huvudtrad} illustrerar programflödet.

\begin{figure}[H]
\centering
\scalebox{0.6}{\input{grafik/huvud-huvudtrad}}
\caption{Flödesschema för huvudtråden} \label{designspec:huvudmodul-huvudtrad}
\end{figure}

\paragraph{Autonomt läge}
\leavevmode
\newline
\newline
Här ska kod för autonom aktivitet hos systemet ligga. Sensorvärdeslistan behöver läsas av varje iteration för att avgöra om systemet befinner sig vid en av- eller upplockningsstation samt om systemet behöver regleras. Om systemet befinner sig vid en upplockningsstation behöver vi läsa av instruktionslistan för att få argument om att styra armen. Om systemet befinner sig vid en avplockningsstation ska systemet låsa sig tills paket är avsläppt.
\newline
Om systemet inte befinner sig vid en av- eller upplockningsstation skickas värden till styrenheten att köra roboten framåt. Dessa värden beräknas i slutet av programmet med regleringsalgoritmen utifrån linjesensordata från sensorvärdeslistan.

\paragraph{Manuellt läge}
\leavevmode
\newline
\newline
I manuellt läge läses argumenten från instruktionslistan av, tolkar dessa och skickar vidare lämpliga värden till styrenheten.

\subsubsection{Sensortråden}
Sensortrådens uppgift är att uppdatera sensorvärdeslistan med fräscha värden från sensorenheten.

\subsubsection{PC-tråden}
PC-tråden delar upp strängen med data från PC-enheten och lagrar instruktionerna och deras argument i instruktionslistan.

\subsection{Reglering}
För att undvika att roboten hackar sig fram och istället rör sig med smidiga rörelser så ska motorns styrsignaler regleras med en tidsdiskret PD-regulator. En PID regulator är onödig då vi inte har något stående fel i teorin. Om roboten är mitt på linjen och står rakt så kommer felet vara noll. I praktiken kommer det inte vara så då roboten troligtvis inte kommer stå rakt på linjen och motorerna kommer säkert åka i olika hastigheter. Dock så blir en PD-regulator tillräckligt bra. En annan anledning till att inte ha med I-delen är att det är svårt att implementera det bra i ett tidsdiskret system.
\begin{itemize}
\item $e[n]$ Är felet vid sampling n
\item $e[n-1]$ Är felet vid sampling n-1
\item $u[n]$ är styrsignalen till motorerna
\item $K_{P}$ är en konstant för den proportionella delen av regleringen
\item $K_{D}$ är en konstant för den deriverande delen av regleringen
\item $\Delta T$ är tiden mellan sampling n och n-1

\end{itemize}
 $$ u[n] = K_P*e[n] + K_D\frac{e[n]-e[n-1]}{\Delta T}$$
 Roboten är begränsade av hårdvaran hur ofta den kan läsa av data från reflexsensorna. En rimlig uppdateringsfrekvens skulle kunna vara 25 Hz\cite{handledare}. Så $\Delta T$ kommer vara 40ms.

 \subsection{Styrning av arm}
 Omvandlingen från x,y,z koordinater till vinklar för armens servon kommer att ske i huvudenheten då det är en krävande operation. Tanken är att när roboten stannar vid en station så ska användaren kunna styra robotens arm med en joystick eller från tangentbordet. När användaren beordrar armen att röra sig framåt så ska armen röra sig i positiv y-led i förhållande till roboten. När användaren beodrar roboten att styra åt höger så ska armens ändpunkt röra sig i positivt x-led i förhållande till roboten.
 \newline

 \begin{figure}[H]
 \centerline{\input{grafik/huvud-armkoordinatsystem}}
 \caption{Armens koordinatsystem i förhållande till roboten. Upp är robotens färdriktning}
 \end{figure}

 \begin{figure}[H]
\centerline{% Graphic for TeX using PGF
% Title: /home/martin/Diagram1.dia
% Creator: Dia v0.97.2
% CreationDate: Sun Oct 12 20:23:47 2014
% For: martin
% \usepackage{tikz}
% The following commands are not supported in PSTricks at present
% We define them conditionally, so when they are implemented,
% this pgf file will use them.
\ifx\du\undefined
  \newlength{\du}
\fi
\setlength{\du}{15\unitlength}
\begin{tikzpicture}
\pgftransformxscale{1.000000}
\pgftransformyscale{-1.000000}
\definecolor{dialinecolor}{rgb}{0.000000, 0.000000, 0.000000}
\pgfsetstrokecolor{dialinecolor}
\definecolor{dialinecolor}{rgb}{1.000000, 1.000000, 1.000000}
\pgfsetfillcolor{dialinecolor}
\pgfsetlinewidth{0.100000\du}
\pgfsetdash{}{0pt}
\pgfsetdash{}{0pt}
\pgfsetbuttcap
{
\definecolor{dialinecolor}{rgb}{0.000000, 0.000000, 0.000000}
\pgfsetfillcolor{dialinecolor}
% was here!!!
\definecolor{dialinecolor}{rgb}{0.000000, 0.000000, 0.000000}
\pgfsetstrokecolor{dialinecolor}
\draw (3.650000\du,5.762500\du)--(7.900000\du,-1.587500\du);
}
\pgfsetlinewidth{0.100000\du}
\pgfsetdash{}{0pt}
\pgfsetdash{}{0pt}
\pgfsetbuttcap
{
\definecolor{dialinecolor}{rgb}{0.000000, 0.000000, 0.000000}
\pgfsetfillcolor{dialinecolor}
% was here!!!
\definecolor{dialinecolor}{rgb}{0.000000, 0.000000, 0.000000}
\pgfsetstrokecolor{dialinecolor}
\draw (7.850000\du,-1.537500\du)--(17.900000\du,-1.487500\du);
}
\pgfsetlinewidth{0.100000\du}
\pgfsetdash{}{0pt}
\pgfsetdash{}{0pt}
\pgfsetbuttcap
{
\definecolor{dialinecolor}{rgb}{0.000000, 0.000000, 0.000000}
\pgfsetfillcolor{dialinecolor}
% was here!!!
\definecolor{dialinecolor}{rgb}{0.000000, 0.000000, 0.000000}
\pgfsetstrokecolor{dialinecolor}
\draw (17.850000\du,-1.437500\du)--(20.900000\du,2.962500\du);
}
\pgfsetlinewidth{0.100000\du}
\pgfsetdash{{1.000000\du}{1.000000\du}}{0\du}
\pgfsetdash{{1.000000\du}{1.000000\du}}{0\du}
\pgfsetbuttcap
{
\definecolor{dialinecolor}{rgb}{0.000000, 0.000000, 0.000000}
\pgfsetfillcolor{dialinecolor}
% was here!!!
\definecolor{dialinecolor}{rgb}{0.000000, 0.000000, 0.000000}
\pgfsetstrokecolor{dialinecolor}
\draw (3.600000\du,6.162500\du)--(3.600000\du,-7.087500\du);
}
\pgfsetlinewidth{0.100000\du}
\pgfsetdash{{1.000000\du}{1.000000\du}}{0\du}
\pgfsetdash{{1.000000\du}{1.000000\du}}{0\du}
\pgfsetbuttcap
{
\definecolor{dialinecolor}{rgb}{0.000000, 0.000000, 0.000000}
\pgfsetfillcolor{dialinecolor}
% was here!!!
\definecolor{dialinecolor}{rgb}{0.000000, 0.000000, 0.000000}
\pgfsetstrokecolor{dialinecolor}
\draw (3.700000\du,6.212500\du)--(27.500000\du,6.062500\du);
}
\pgfsetlinewidth{0.100000\du}
\pgfsetdash{}{0pt}
\pgfsetdash{}{0pt}
\pgfsetbuttcap
{
\definecolor{dialinecolor}{rgb}{0.000000, 0.000000, 0.000000}
\pgfsetfillcolor{dialinecolor}
% was here!!!
\definecolor{dialinecolor}{rgb}{0.000000, 0.000000, 0.000000}
\pgfsetstrokecolor{dialinecolor}
\pgfpathmoveto{\pgfpoint{6.049937\du}{6.212702\du}}
\pgfpatharc{18}{-65}{2.110604\du and 2.110604\du}
\pgfusepath{stroke}
}
\pgfsetlinewidth{0.100000\du}
\pgfsetdash{}{0pt}
\pgfsetdash{}{0pt}
\pgfsetbuttcap
{
\definecolor{dialinecolor}{rgb}{0.000000, 0.000000, 0.000000}
\pgfsetfillcolor{dialinecolor}
% was here!!!
\definecolor{dialinecolor}{rgb}{0.000000, 0.000000, 0.000000}
\pgfsetstrokecolor{dialinecolor}
\pgfpathmoveto{\pgfpoint{6.949854\du}{0.062444\du}}
\pgfpatharc{112}{9}{2.148827\du and 2.148827\du}
\pgfusepath{stroke}
}
\pgfsetlinewidth{0.100000\du}
\pgfsetdash{}{0pt}
\pgfsetdash{}{0pt}
\pgfsetbuttcap
{
\definecolor{dialinecolor}{rgb}{0.000000, 0.000000, 0.000000}
\pgfsetfillcolor{dialinecolor}
% was here!!!
\definecolor{dialinecolor}{rgb}{0.000000, 0.000000, 0.000000}
\pgfsetstrokecolor{dialinecolor}
\pgfpathmoveto{\pgfpoint{16.550022\du}{-1.537549\du}}
\pgfpatharc{205}{42}{1.170625\du and 1.170625\du}
\pgfusepath{stroke}
}
\pgfsetlinewidth{0.100000\du}
\pgfsetdash{{1.000000\du}{1.000000\du}}{0\du}
\pgfsetdash{{1.000000\du}{1.000000\du}}{0\du}
\pgfsetbuttcap
{
\definecolor{dialinecolor}{rgb}{0.000000, 0.000000, 0.000000}
\pgfsetfillcolor{dialinecolor}
% was here!!!
\definecolor{dialinecolor}{rgb}{0.000000, 0.000000, 0.000000}
\pgfsetstrokecolor{dialinecolor}
\draw (18.200000\du,2.962500\du)--(27.950000\du,2.912500\du);
}
\pgfsetlinewidth{0.100000\du}
\pgfsetdash{}{0pt}
\pgfsetdash{}{0pt}
\pgfsetbuttcap
{
\definecolor{dialinecolor}{rgb}{0.000000, 0.000000, 0.000000}
\pgfsetfillcolor{dialinecolor}
% was here!!!
\definecolor{dialinecolor}{rgb}{0.000000, 0.000000, 0.000000}
\pgfsetstrokecolor{dialinecolor}
\pgfpathmoveto{\pgfpoint{19.950016\du}{1.812506\du}}
\pgfpatharc{292}{159}{0.849480\du and 0.849480\du}
\pgfusepath{stroke}
}
% setfont left to latex
\definecolor{dialinecolor}{rgb}{0.000000, 0.000000, 0.000000}
\pgfsetstrokecolor{dialinecolor}
\node[anchor=west] at (0.500000\du,6.112500\du){P0=(0,0)};
% setfont left to latex
\definecolor{dialinecolor}{rgb}{0.000000, 0.000000, 0.000000}
\pgfsetstrokecolor{dialinecolor}
\node[anchor=west] at (5.650000\du,4.062500\du){A1};
% setfont left to latex
\definecolor{dialinecolor}{rgb}{0.000000, 0.000000, 0.000000}
\pgfsetstrokecolor{dialinecolor}
\node[anchor=west] at (8.900000\du,0.862500\du){A2};
% setfont left to latex
\definecolor{dialinecolor}{rgb}{0.000000, 0.000000, 0.000000}
\pgfsetstrokecolor{dialinecolor}
\node[anchor=west] at (16.350000\du,0.762500\du){A3};
% setfont left to latex
\definecolor{dialinecolor}{rgb}{0.000000, 0.000000, 0.000000}
\pgfsetstrokecolor{dialinecolor}
\node[anchor=west] at (14.800000\du,2.562500\du){};
% setfont left to latex
\definecolor{dialinecolor}{rgb}{0.000000, 0.000000, 0.000000}
\pgfsetstrokecolor{dialinecolor}
\node[anchor=west] at (18.150000\du,1.812500\du){};
% setfont left to latex
\definecolor{dialinecolor}{rgb}{0.000000, 0.000000, 0.000000}
\pgfsetstrokecolor{dialinecolor}
\node[anchor=west] at (17.850000\du,2.062500\du){A4};
% setfont left to latex
\definecolor{dialinecolor}{rgb}{0.000000, 0.000000, 0.000000}
\pgfsetstrokecolor{dialinecolor}
\node[anchor=west] at (4.850000\du,1.162500\du){L1};
% setfont left to latex
\definecolor{dialinecolor}{rgb}{0.000000, 0.000000, 0.000000}
\pgfsetstrokecolor{dialinecolor}
\node[anchor=west] at (12.600000\du,-2.287500\du){L2};
% setfont left to latex
\definecolor{dialinecolor}{rgb}{0.000000, 0.000000, 0.000000}
\pgfsetstrokecolor{dialinecolor}
\node[anchor=west] at (20.050000\du,0.362500\du){L3};
% setfont left to latex
\definecolor{dialinecolor}{rgb}{0.000000, 0.000000, 0.000000}
\pgfsetstrokecolor{dialinecolor}
\node[anchor=west] at (7.100000\du,-2.337500\du){P1=(W1,Z1)};
% setfont left to latex
\definecolor{dialinecolor}{rgb}{0.000000, 0.000000, 0.000000}
\pgfsetstrokecolor{dialinecolor}
\node[anchor=west] at (18.175000\du,-2.337500\du){P2=(W2,Z2)};
% setfont left to latex
\definecolor{dialinecolor}{rgb}{0.000000, 0.000000, 0.000000}
\pgfsetstrokecolor{dialinecolor}
\node[anchor=west] at (20.825000\du,3.912500\du){P3=(W,Z)};
% setfont left to latex
\definecolor{dialinecolor}{rgb}{0.000000, 0.000000, 0.000000}
\pgfsetstrokecolor{dialinecolor}
\node[anchor=west] at (20.925000\du,-2.637500\du){};
\pgfsetlinewidth{0.100000\du}
\pgfsetdash{}{0pt}
\pgfsetdash{}{0pt}
\pgfsetbuttcap
{
\definecolor{dialinecolor}{rgb}{0.000000, 0.000000, 0.000000}
\pgfsetfillcolor{dialinecolor}
% was here!!!
\definecolor{dialinecolor}{rgb}{0.000000, 0.000000, 0.000000}
\pgfsetstrokecolor{dialinecolor}
\draw (3.675000\du,6.212500\du)--(17.725000\du,-1.487500\du);
}
% setfont left to latex
\definecolor{dialinecolor}{rgb}{0.000000, 0.000000, 0.000000}
\pgfsetstrokecolor{dialinecolor}
\node[anchor=west] at (11.775000\du,2.762500\du){HL};
% setfont left to latex
\definecolor{dialinecolor}{rgb}{0.000000, 0.000000, 0.000000}
\pgfsetstrokecolor{dialinecolor}
\node[anchor=west] at (3.225000\du,-8.087500\du){z};
% setfont left to latex
\definecolor{dialinecolor}{rgb}{0.000000, 0.000000, 0.000000}
\pgfsetstrokecolor{dialinecolor}
\node[anchor=west] at (27.125000\du,6.062500\du){w};
\pgfsetlinewidth{0.100000\du}
\pgfsetdash{}{0pt}
\pgfsetdash{}{0pt}
\pgfsetbuttcap
{
\definecolor{dialinecolor}{rgb}{0.000000, 0.000000, 0.000000}
\pgfsetfillcolor{dialinecolor}
% was here!!!
\definecolor{dialinecolor}{rgb}{0.000000, 0.000000, 0.000000}
\pgfsetstrokecolor{dialinecolor}
\pgfpathmoveto{\pgfpoint{8.425000\du}{6.212503\du}}
\pgfpatharc{358}{321}{3.739603\du and 3.739603\du}
\pgfusepath{stroke}
}
% setfont left to latex
\definecolor{dialinecolor}{rgb}{0.000000, 0.000000, 0.000000}
\pgfsetstrokecolor{dialinecolor}
\node[anchor=west] at (8.825000\du,4.762500\du){HA};
\end{tikzpicture}
}
\caption{Armens vinklar}
\end{figure}

%\newline
 För att detta ska vara möjligt så behöver vi använda lite trionometri. Omvandlingen sker på följande sätt:
Antag att roboten start i: $$X,Y,Z,GR=0,0,0,\pi/2$$
Där GR(GripRadian) är vilken vinkel gripklon har till z-axeln.\newline
Vi omvandlar:$$ X,Y,Z,GR\rightarrow A,W,Z,GR$$
Där A är armens vinkel mot positiva x-axeln och W är armens utsträckning.
$$A=tan^-1(\dfrac{Y}{X}) $$
$$W=\sqrt{X^2+Y^2}$$
Genom denna omvandlingen så kan vi göra om detta till ett 2-dimensionellt problem istället för ett 3-dimentionellt problem. Vi får in X,Y,Z,GR och ställer in robotens vinkel mot x-axelns positiva del och sedan ser vi till att roboten bara har rätt utsträckning och höjd.
$$P_4=(W,Z)$$
$$P_0=(0,0)$$
$$W_2=W-cos(GR)*L_3$$
$$Z_2=Z-sin(GR)*L_3$$
$$H_L=\sqrt{W_2^2+Z_2^2}$$
$$H_A=tan^-1(\dfrac{Z_2}{W_2})$$
Cosinussatsen ger:
$$A_1=cos^-1(\dfrac{L_1^2+H_L^2-L_2^2}{2*L_1*H_L})+H_A$$
$$W_1=cos(A_1)*L_1$$
$$Z_1=sin(A_1)*L_1$$
$$A_2=tan^-1(\dfrac{Z_2-Z_1}{W_2-W_1})-A_1$$
$$A_3=GR-D_2-D_3$$

Då har vi alla vinklar vi behöver. Sedan är det upp till styrenheten att omvandla dessa vinklar till positioner på servona på armen.
Beräkningarna har simulerats i ett eget python-program (figur \ref{designspec-huvud-sim-arm}).
\begin{figure}[H]
\center
\includegraphics[scale=0.55]{grafik/huvud-simulator}
\caption{Simulerad arm} \label{designspec-huvud-sim-arm}
\end{figure}

\subsection{Komponenter}
Följande komponenter är nödvändiga för konstruktion av huvudmodulen.\\
\begin{tabularx}{\textwidth}{| l | X |}
	\hline
	{\textbf{Komponent}} & {\textbf{Tillgänglighet}} \\\hline
	{Beagleboard-xm} & {Tillgänglig} \\\hline
	{Blåtandsdongel Belkin f8t016} & {Tillgänglig} \\\hline
	{Minneskort} & {Skall beställas} \\\hline
	{Nivåomvandlare Ti TXB0108} & {Skall beställas. Förslagsvis från electrokit\cite{nivaomvandlare}} \\\hline
	{Wifidongel} & {Tillgänglig} \\\hline
\end{tabularx}
