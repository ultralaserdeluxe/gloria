\section{Översikt av system}
Plattformen består av fyra enheter. 
\begin{itemize}
\item{PC}
\item{Huvudenhet}
\item{Sensorenhet}
\item{Styrenhet}
\end{itemize}

PC är framförallt ett användargränssnitt som gör det enkelt för användaren att styra roboten och framförallt robotarmen. Huvudenheten står för huvuddelen av alla beräkningar roboten behöver göra så som regleringsalgoritm för linjeföljaren och koordinatkonvertering för armen. Sensorenheten innehåller den funktionalitet som krävs för att driva alla systemets sensorer medans styrenheten innehåller den funktionalitet som krävs för att köra motorer och servon.

\subsection{Kommunikation}
Styrenheten och sensorenheten kommer att kommuncera med huvudenheten över varsin SPI-buss. Kommunikation mellan huvudmodulen och PC kommer att ske över Bluetooth. Förslaget är att sätta upp ett PAN (PERSONAL AREA NETWORK) mellan PC:n och huvudmodulen. Detta möjliggör kommunikation över TCP/IP protokollet.

\begin{figure}[h]
\center
\scalebox{0.6}{% Graphic for TeX using PGF
% Title: /home/hannes/GitHub/TSEA29/dokumentation/designspecifikation/FLOW1.dia
% Creator: Dia v0.97.2
% CreationDate: Thu Oct  9 16:44:50 2014
% For: hannes
% \usepackage{tikz}
% The following commands are not supported in PSTricks at present
% We define them conditionally, so when they are implemented,
% this pgf file will use them.
\ifx\du\undefined
  \newlength{\du}
\fi
\setlength{\du}{15\unitlength}
\begin{tikzpicture}
\pgftransformxscale{1.000000}
\pgftransformyscale{-1.000000}
\definecolor{dialinecolor}{rgb}{0.000000, 0.000000, 0.000000}
\pgfsetstrokecolor{dialinecolor}
\definecolor{dialinecolor}{rgb}{1.000000, 1.000000, 1.000000}
\pgfsetfillcolor{dialinecolor}
\definecolor{dialinecolor}{rgb}{1.000000, 1.000000, 1.000000}
\pgfsetfillcolor{dialinecolor}
\fill (2.704150\du,11.806300\du)--(2.704150\du,18.306300\du)--(12.454150\du,18.306300\du)--(12.454150\du,11.806300\du)--cycle;
\pgfsetlinewidth{0.100000\du}
\pgfsetdash{}{0pt}
\pgfsetdash{}{0pt}
\pgfsetmiterjoin
\definecolor{dialinecolor}{rgb}{0.000000, 0.000000, 0.000000}
\pgfsetstrokecolor{dialinecolor}
\draw (2.704150\du,11.806300\du)--(2.704150\du,18.306300\du)--(12.454150\du,18.306300\du)--(12.454150\du,11.806300\du)--cycle;
% setfont left to latex
\definecolor{dialinecolor}{rgb}{0.000000, 0.000000, 0.000000}
\pgfsetstrokecolor{dialinecolor}
\node at (7.579150\du,15.251300\du){HUVUDENHET};
\definecolor{dialinecolor}{rgb}{1.000000, 1.000000, 1.000000}
\pgfsetfillcolor{dialinecolor}
\fill (-6.345850\du,13.456300\du)--(-6.345850\du,16.806300\du)--(-1.645850\du,16.806300\du)--(-1.645850\du,13.456300\du)--cycle;
\pgfsetlinewidth{0.100000\du}
\pgfsetdash{}{0pt}
\pgfsetdash{}{0pt}
\pgfsetmiterjoin
\definecolor{dialinecolor}{rgb}{0.000000, 0.000000, 0.000000}
\pgfsetstrokecolor{dialinecolor}
\draw (-6.345850\du,13.456300\du)--(-6.345850\du,16.806300\du)--(-1.645850\du,16.806300\du)--(-1.645850\du,13.456300\du)--cycle;
% setfont left to latex
\definecolor{dialinecolor}{rgb}{0.000000, 0.000000, 0.000000}
\pgfsetstrokecolor{dialinecolor}
\node at (-3.995850\du,15.326300\du){SPI};
\definecolor{dialinecolor}{rgb}{1.000000, 1.000000, 1.000000}
\pgfsetfillcolor{dialinecolor}
\fill (16.254200\du,13.256300\du)--(16.254200\du,16.756300\du)--(21.104200\du,16.756300\du)--(21.104200\du,13.256300\du)--cycle;
\pgfsetlinewidth{0.100000\du}
\pgfsetdash{}{0pt}
\pgfsetdash{}{0pt}
\pgfsetmiterjoin
\definecolor{dialinecolor}{rgb}{0.000000, 0.000000, 0.000000}
\pgfsetstrokecolor{dialinecolor}
\draw (16.254200\du,13.256300\du)--(16.254200\du,16.756300\du)--(21.104200\du,16.756300\du)--(21.104200\du,13.256300\du)--cycle;
% setfont left to latex
\definecolor{dialinecolor}{rgb}{0.000000, 0.000000, 0.000000}
\pgfsetstrokecolor{dialinecolor}
\node at (18.679200\du,15.201300\du){SPI};
\pgfsetlinewidth{0.100000\du}
\pgfsetdash{}{0pt}
\pgfsetdash{}{0pt}
\pgfsetbuttcap
{
\definecolor{dialinecolor}{rgb}{0.000000, 0.000000, 0.000000}
\pgfsetfillcolor{dialinecolor}
% was here!!!
\pgfsetarrowsstart{to}
\pgfsetarrowsend{to}
\definecolor{dialinecolor}{rgb}{0.000000, 0.000000, 0.000000}
\pgfsetstrokecolor{dialinecolor}
\draw (-1.595931\du,15.115750\du)--(2.654971\du,15.088206\du);
}
\pgfsetlinewidth{0.100000\du}
\pgfsetdash{}{0pt}
\pgfsetdash{}{0pt}
\pgfsetbuttcap
{
\definecolor{dialinecolor}{rgb}{0.000000, 0.000000, 0.000000}
\pgfsetfillcolor{dialinecolor}
% was here!!!
\pgfsetarrowsstart{to}
\pgfsetarrowsend{to}
\definecolor{dialinecolor}{rgb}{0.000000, 0.000000, 0.000000}
\pgfsetstrokecolor{dialinecolor}
\draw (12.503849\du,15.034117\du)--(16.204317\du,15.017448\du);
}
\definecolor{dialinecolor}{rgb}{1.000000, 1.000000, 1.000000}
\pgfsetfillcolor{dialinecolor}
\fill (-8.145850\du,19.856300\du)--(-8.145850\du,25.956300\du)--(0.354150\du,25.956300\du)--(0.354150\du,19.856300\du)--cycle;
\pgfsetlinewidth{0.100000\du}
\pgfsetdash{}{0pt}
\pgfsetdash{}{0pt}
\pgfsetmiterjoin
\definecolor{dialinecolor}{rgb}{0.000000, 0.000000, 0.000000}
\pgfsetstrokecolor{dialinecolor}
\draw (-8.145850\du,19.856300\du)--(-8.145850\du,25.956300\du)--(0.354150\du,25.956300\du)--(0.354150\du,19.856300\du)--cycle;
% setfont left to latex
\definecolor{dialinecolor}{rgb}{0.000000, 0.000000, 0.000000}
\pgfsetstrokecolor{dialinecolor}
\node at (-3.895850\du,23.101300\du){STYRENHET};
\definecolor{dialinecolor}{rgb}{1.000000, 1.000000, 1.000000}
\pgfsetfillcolor{dialinecolor}
\fill (14.404200\du,19.406300\du)--(14.404200\du,25.606300\du)--(22.904200\du,25.606300\du)--(22.904200\du,19.406300\du)--cycle;
\pgfsetlinewidth{0.100000\du}
\pgfsetdash{}{0pt}
\pgfsetdash{}{0pt}
\pgfsetmiterjoin
\definecolor{dialinecolor}{rgb}{0.000000, 0.000000, 0.000000}
\pgfsetstrokecolor{dialinecolor}
\draw (14.404200\du,19.406300\du)--(14.404200\du,25.606300\du)--(22.904200\du,25.606300\du)--(22.904200\du,19.406300\du)--cycle;
% setfont left to latex
\definecolor{dialinecolor}{rgb}{0.000000, 0.000000, 0.000000}
\pgfsetstrokecolor{dialinecolor}
\node at (18.654200\du,22.701300\du){SENSORENHET};
\pgfsetlinewidth{0.100000\du}
\pgfsetdash{}{0pt}
\pgfsetdash{}{0pt}
\pgfsetbuttcap
{
\definecolor{dialinecolor}{rgb}{0.000000, 0.000000, 0.000000}
\pgfsetfillcolor{dialinecolor}
% was here!!!
\pgfsetarrowsstart{to}
\pgfsetarrowsend{to}
\definecolor{dialinecolor}{rgb}{0.000000, 0.000000, 0.000000}
\pgfsetstrokecolor{dialinecolor}
\draw (-3.973670\du,16.855809\du)--(-3.935724\du,19.806076\du);
}
\pgfsetlinewidth{0.100000\du}
\pgfsetdash{}{0pt}
\pgfsetdash{}{0pt}
\pgfsetbuttcap
{
\definecolor{dialinecolor}{rgb}{0.000000, 0.000000, 0.000000}
\pgfsetfillcolor{dialinecolor}
% was here!!!
\pgfsetarrowsstart{to}
\pgfsetarrowsend{to}
\definecolor{dialinecolor}{rgb}{0.000000, 0.000000, 0.000000}
\pgfsetstrokecolor{dialinecolor}
\draw (18.673199\du,16.806685\du)--(18.664700\du,19.356428\du);
}
\definecolor{dialinecolor}{rgb}{1.000000, 1.000000, 1.000000}
\pgfsetfillcolor{dialinecolor}
\fill (5.729150\du,6.781310\du)--(5.729150\du,9.481310\du)--(9.379150\du,9.481310\du)--(9.379150\du,6.781310\du)--cycle;
\pgfsetlinewidth{0.100000\du}
\pgfsetdash{}{0pt}
\pgfsetdash{}{0pt}
\pgfsetmiterjoin
\definecolor{dialinecolor}{rgb}{0.000000, 0.000000, 0.000000}
\pgfsetstrokecolor{dialinecolor}
\draw (5.729150\du,6.781310\du)--(5.729150\du,9.481310\du)--(9.379150\du,9.481310\du)--(9.379150\du,6.781310\du)--cycle;
% setfont left to latex
\definecolor{dialinecolor}{rgb}{0.000000, 0.000000, 0.000000}
\pgfsetstrokecolor{dialinecolor}
\node at (7.554150\du,7.926310\du){BT};
% setfont left to latex
\definecolor{dialinecolor}{rgb}{0.000000, 0.000000, 0.000000}
\pgfsetstrokecolor{dialinecolor}
\node at (7.554150\du,8.726310\du){USB};
\pgfsetlinewidth{0.100000\du}
\pgfsetdash{}{0pt}
\pgfsetdash{}{0pt}
\pgfsetbuttcap
{
\definecolor{dialinecolor}{rgb}{0.000000, 0.000000, 0.000000}
\pgfsetfillcolor{dialinecolor}
% was here!!!
\pgfsetarrowsstart{to}
\pgfsetarrowsend{to}
\definecolor{dialinecolor}{rgb}{0.000000, 0.000000, 0.000000}
\pgfsetstrokecolor{dialinecolor}
\draw (7.559205\du,9.531609\du)--(7.567334\du,11.783160\du);
}
\definecolor{dialinecolor}{rgb}{1.000000, 1.000000, 1.000000}
\pgfsetfillcolor{dialinecolor}
\fill (4.921370\du,-4.547630\du)--(4.921370\du,-1.269208\du)--(10.224661\du,-1.269208\du)--(10.224661\du,-4.547630\du)--cycle;
\pgfsetlinewidth{0.100000\du}
\pgfsetdash{}{0pt}
\pgfsetdash{}{0pt}
\pgfsetmiterjoin
\definecolor{dialinecolor}{rgb}{0.000000, 0.000000, 0.000000}
\pgfsetstrokecolor{dialinecolor}
\draw (4.921370\du,-4.547630\du)--(4.921370\du,-1.269208\du)--(10.224661\du,-1.269208\du)--(10.224661\du,-4.547630\du)--cycle;
% setfont left to latex
\definecolor{dialinecolor}{rgb}{0.000000, 0.000000, 0.000000}
\pgfsetstrokecolor{dialinecolor}
\node at (7.573016\du,-2.713419\du){PC};
\definecolor{dialinecolor}{rgb}{1.000000, 1.000000, 1.000000}
\pgfsetfillcolor{dialinecolor}
\fill (5.938090\du,1.451670\du)--(5.938090\du,4.151670\du)--(9.155420\du,4.151670\du)--(9.155420\du,1.451670\du)--cycle;
\pgfsetlinewidth{0.100000\du}
\pgfsetdash{}{0pt}
\pgfsetdash{}{0pt}
\pgfsetmiterjoin
\definecolor{dialinecolor}{rgb}{0.000000, 0.000000, 0.000000}
\pgfsetstrokecolor{dialinecolor}
\draw (5.938090\du,1.451670\du)--(5.938090\du,4.151670\du)--(9.155420\du,4.151670\du)--(9.155420\du,1.451670\du)--cycle;
% setfont left to latex
\definecolor{dialinecolor}{rgb}{0.000000, 0.000000, 0.000000}
\pgfsetstrokecolor{dialinecolor}
\node at (7.546755\du,2.596670\du){USB};
% setfont left to latex
\definecolor{dialinecolor}{rgb}{0.000000, 0.000000, 0.000000}
\pgfsetstrokecolor{dialinecolor}
\node at (7.546755\du,3.396670\du){BT};
\pgfsetlinewidth{0.100000\du}
\pgfsetdash{}{0pt}
\pgfsetdash{}{0pt}
\pgfsetbuttcap
{
\definecolor{dialinecolor}{rgb}{0.000000, 0.000000, 0.000000}
\pgfsetfillcolor{dialinecolor}
% was here!!!
\pgfsetarrowsstart{to}
\pgfsetarrowsend{to}
\definecolor{dialinecolor}{rgb}{0.000000, 0.000000, 0.000000}
\pgfsetstrokecolor{dialinecolor}
\draw (7.565258\du,-1.221601\du)--(7.553194\du,1.401681\du);
}
\pgfsetlinewidth{0.100000\du}
\pgfsetdash{{\pgflinewidth}{0.200000\du}}{0cm}
\pgfsetdash{{\pgflinewidth}{0.200000\du}}{0cm}
\pgfsetbuttcap
{
\definecolor{dialinecolor}{rgb}{0.000000, 0.000000, 0.000000}
\pgfsetfillcolor{dialinecolor}
% was here!!!
\pgfsetarrowsstart{to}
\pgfsetarrowsend{to}
\definecolor{dialinecolor}{rgb}{0.000000, 0.000000, 0.000000}
\pgfsetstrokecolor{dialinecolor}
\draw (7.548694\du,4.199139\du)--(7.552211\du,6.733841\du);
}
\end{tikzpicture}
}
\caption{Flödesschema över kommunikationskanalerna i systemet.}
\end{figure}

\subsection{Uppgraderbarhet}
Kommunikation mellan huvudmodulen och PC:n kommer att ske över bluetooth och kommer att användas för att sätta upp ett PAN. Över denna sätts en TCP/IP anslutning upp och data skickas över en Python-socket vilket leder till att bluetooth kan bytas ut mot till exempel WIFI eller en ethernetkabel.
