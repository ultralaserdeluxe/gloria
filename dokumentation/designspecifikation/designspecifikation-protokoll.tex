\section{Protokoll för kommunikation}

\subsection{Kommunikation mellan PC och huvudmodul}

\subsubsection{Hårdvara}
Kommunikation mellan PC och huvudmodul sker via blåtand.

\subsubsection{Kommandon}
Kommandon ges som en sträng på formen "command1=arg1,arg...;command2=arg1,arg..." med ett godtyckligt antal kommandon. \\

\begin{table}[h]
	\centering
		\begin{tabularx}{\textwidth}{| l | l | X |}
			\hline
			\textbf{Kommando} & \textbf{Argument} & \textbf{Beskrivning} \\
			\hline
			{motor} & {L,R} & {L och R anger hastigheten på det vänstra, respektive högra hjulparet} \\
			\hline
			{arm} & {X,Y,Z,P,G} & {X,Y,Z är koordinaten i rummet dit armens "hand" skall röra sig, armens fundament är origo. P anger handens vinkel i förhållande till Z-axeln och G anger avståndet mellan handens fingrar.} \\
			\hline
			{calibrate} & {} & {Begär kalibrering av robotens sensorer} \\
			\hline
			{status} & {} & {Begär statusrapport från robot \todo{Format på svar?}} \\
			\hline
			{automotor} & {M} & {Anger om robotens motorer skall styras från PC (0) eller autonomt (1)} \\
			\hline
			{autoarm} & {M} & {Anger om robotarmen skall styras från PC (0) eller autonomt (1)} \\
			\hline
			{start} & {} & {Initiera körning} \\
			\hline
		\end{tabularx}
	\caption{Kommandon från PC till huvudmodul} \label{protokoll:pc-huvud}
\end{table}
\todo{Skriv om kommandobeskrivningar, blandar presens, futurum osv} \\
\todo{Högre/lägre värden på t.ex X eller P gör vad, precis?}\\
\todo{Illustrera koordinatsystem m. bild}

\subsection{Kommunikation mellan huvudmodul och sensormodul}
Kommunikation mellan huvudmodul och sensormodul sker via en SPI-buss. 

\subsubsection{Hårdvara}
Kommunikation mellan huvudmodul och sensormodul sker via en SPI-buss. Kommandon från huvudmodulen ges av en byte. De fyra första bitarna ger vilket kommando som skall utföras. De sista fyra ger vilken sensor som gäller. 
Sensorenheten svarar med endast data. Två bytes för vardera linjesensor med kalibrerad data. En byte för vardera avståndssensor bestående av de 8 MSB av de 10 bitar vi får från sensorn. I fallet med kommando läs data från alla sensorer, skickas sensorernas data seriellt i följande ordning: 1. Linjesensor 1, 2: Linjesensor 2, 3: Avståndssensor Höger, 4: Avståndssensor Vänster.

\subsubsection{Kommandon}

\begin{table}[h]
	\centering
		\begin{tabularx}{\textwidth}{| l | l | X |}
			\hline
			\textbf{Kommando} & \textbf{Argument} & \textbf{Beskrivning} \\
			\hline
			{0000} & {A} & {Returnera sensordata för A} \\
			\hline
			{0001} & {A} & {Kalibrera sensor A} \\
			\hline
		\end{tabularx}
	\caption{Kommandon från huvudmodul till sensormodul} \label{protokoll:huvud-sensor}
\end{table}

\begin{table}[h]
	\centering
		\begin{tabularx}{\textwidth}{| l | X |}
			\hline
			\textbf{Adress} & \textbf{Beskrivning} \\
			\hline
			{0000} & {Linjesensor Fram} \\
			\hline
			{0001} & {Linjesensor Bak} \\
			\hline
			{0010} & {Avståndssensor Höger} \\
			\hline
			{0011} & {Avståndssensor Vänster} \\
			\hline
			{1111} & {Adressera samtliga sensorer} \\
			\hline
		\end{tabularx}
	\caption{Adresser för kammandon till motormodul} \label{protokoll:huvud-sensor-adress}
\end{table}

\todo{Format på returdata från sensormodul?} \\

Modulens busy-pinne (optokoplpare) är definierad som 0 när modulen arbetar, 1 när den är ledig

\subsection{Kommunikation mellan huvudmodul och motormodul} \label{protokoll:pc-motor}

\subsubsection{Hårdvara}
Kommunikation mellan huvudmodul och motormodul sker via en SPI-buss. En flagga återkopplar och talar om för huvudmodulen huruvida armen är i rörelse eller inte.

\subsubsection{Kommandon}
Ett kommando består av tre bytes. De första fyra bitarna definierar vilket kommando som skall utföras. De nästkommande fyra vilken enhet det skall utföras av. Därefter följer två databytes. 

\begin{table}[h]
	\centering
		\begin{tabularx}{\textwidth}{| l | l | X |}
			\hline
			\textbf{Kommando} & \textbf{Argument} & \textbf{Beskrivning} \\
			\hline
			{0000} & {} & {Stoppa samtliga servon och motorer} \\
			\hline
			{0001} & {A, D} & {Sätt register A till D} \\
			\hline
			{0010} & {A} & {Utför givna kommandon för A} \\
			\hline
		\end{tabularx}
	\caption{Kommandon från huvudmodul till motormodul} \label{protokoll:pc-motor-tabell}
\end{table}

\begin{table}[h]
	\centering
		\begin{tabularx}{\textwidth}{| l | X |}
			\hline
			\textbf{Adress} & \textbf{Beskrivning} \\
			\hline
			{0000} & {Höger hjulpar} \\
			\hline
			{0001} & {Vänster hjulpar} \\
			\hline
			{0010} & {Arm axel 1} \\ % Armens bas?
			\hline
			{0100} & {Arm axel 2} \\
			\hline
			{0110} & {Arm axel 3} \\
			\hline
			{1000} & {Arm axel 4} \\
			\hline
			{1011} & {Arm axel 5} \\ % Gripklo?
			\hline
			{1100} & {Samtliga motorer} \\
			\hline
			{1101} & {Samtliga servon} \\
			\hline
			{1111} & {Samtliga motorer och servon} \\
			\hline
		\end{tabularx}
	\caption{Adresser för adressering till motormodul} \label{protokoll:pc-motor-adress-tabell}
\end{table}
\todo{Illustrera axel-nummer med bild?} \\

Det finns en busy-pinne (optokopplare) som är definierad som 0 när armen är upptagen, 1 när den är ledig. \todo{Från huvudmodulen finns en haltflagga som stoppar all körning i modulen.}