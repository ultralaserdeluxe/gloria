\section{Översikt av system}
På plattformen kommer tre moduler att installeras:

\begin{itemize}
\item Huvudmodul
\item Styrmodul
\item Sensormodul
\end{itemize}
\subsection{Kommunikation}
Kommunikationen mellan Huvudmodulen, styrmodulen och sensormodulen kommer förslagsvis att ske över UART. Huvudmodulen kommer isåfall ha två stycken USB-RS232 konverterare som kommer kopplas till varsin modul. Kommunikationen från huvudmodulen till PC kommer att ske över Bluetooth. Förslaget är att sätta upp ett PAN (PERSONAL AREA NETWORK) mellan PC:n och huvudmodulen. Detta möjliggör kommunikation över TCP/IP protokollet.
\newline
\centerline{\includegraphics[scale=0.4]{FLOW1PNG}}
\centerline{Flödesschema över kommunikationen i systemet}
\newline
\newline
\centerline{\includegraphics[scale=0.6]{PC-huvud}}
\centerline{Kommunikation mellan PC och huvudmodulen}
\newline
\newline
\centerline{\includegraphics[scale=0.6]{huvud-sensor}}
\centerline{Kommunikation mellan huvudmodulen och sensormodulen}
\newline
\newline
\centerline{\includegraphics[scale=0.6]{huvud-styr}}
\centerline{Kommunikation mellan huvudmodulen och styrmodulen}

\subsection{Uppgraderbarhet}
Tydliga kommunikationsprotokoll ska existera eftersom det blir enklare att byta ut en modul. Kommunikationen mellan huvudmodulen och övriga moduler sker förslagsvis över UART. Eftersom detta sker med hjälp av USB-UART donglar så finns det möjlighet att lägga till fler moduler i framtiden eftersom Beagleboarden har fyra USB-portar och tre kommer användas i systemet. Behövs fler i framtiden så kan en USB-hubb användas.
\newline
\newline
Kommunikationen mellan huvudmodulen och PC:n kommer förslagsvis att ske över bluetooth och isåfall kommer den användas för att sätta upp ett PAN. Över denna sätts en TCP/IP anslutning upp och data skickas över en Python-socket vilket leder till att bluetooth kan bytas ut mot WIFI eller en ethernetkabel.
