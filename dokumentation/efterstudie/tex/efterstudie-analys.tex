% Hur arbetade vi och hur fungerade det

\section{Analys av arbete och problem}

\subsection{Händelser}
\subsubsection{Förfasen}

%saker som kunde ha varit bättre
\textbf{Kunde ha varit bättre:}
\newline
Det hade varit vettigt att speca aktiviteterna noggrannare så att man vet när de är avklarade, en liten beskrivning till varje aktivitet hade varit bra. Vi skulle kunna ha haft fler personer på varje aktivitet.
\newline

Vi borde ha delat in aktiviteterna i början på varje vecka. Det hade varit både flexiblare och smidigare.
\newline

Vi delade in aktiviteterna för mycket. Vi borde inte skilja så mycket mellan implementation och testning. Dessa två moment borde ingå i samma aktivitet.
\newline

Vi borde ha planerat mer tid för integrationstester av systemet och delsystem, speciellt tidigare i projektet.
\newline 

%saker som var bra från början
\textbf{Bra:}
\newline
Ett bra beslut var att skriva all dokumentation i LateX, och att vi använde Git från första början (LateX är versionshanterat i Git).
\newline

Systemskissen var väl genomtänkt.
\newline

Det var en väldigt bra idé att ha konkreta mötesprotokoll. Dessa följdes dock inte till punkt och pricka, samt att vissa element (såsom att en person pratar i taget under en viss tidsperiod) var inte jättebra i slutändan.

\subsubsection{Underfasen}

%saker som kunde ha varit bättre
\textbf{Kunde ha varit bättre:}
\newline
Vi kunde ha fördelat tiden bättre. Vi stannade av i mitten av projektet i stället för att fortsätta och slutföra utan att behöva stressa i slutet.
\newline

Interfacet skulle ha planerats bättre.
\newline

Vi skulle ha haft full version av Eagle för att skapa ett ordentligt kopplingsschema för kretskortet. Hade vi dessutom vetat att vi skulle ha haft ett kretskort från början så hade vi planerat in aktiviteter (arbetstid) för det.
\newline

%saker som var bra
\textbf{Bra:}
\newline
Roboten blev godkänd i tid.
\newline

Designspecen blev bra. Alla var med och bestämde delar av hur systemet skulle byggas vilket ledde till att alla var ganska medvetna om hur implementationen skulle göras från alla håll.

\subsubsection{Efterfasen}

%saker som var bra
\textbf{Bra:}
\newline
Vi fick ingen restlista när vi hade vår BP5 (kravgenomgång).

\subsection{Samarbete}

%saker som kunde ha varit bättre
\textbf{Kunde ha varit bättre:}
\newline
Möten utlystes lite sent. Vi hade problem att komma överens om när exakt vi skulle hålla mötena på dagarna.
\newline

Vi skulle kunna ha haft en gruppmail där alla får kopior på alla mail.
\newline

%saker som var bra
\textbf{Bra:}
\newline
Kommunikationen har funkat bra över lag. Ansvaret har axlats av projektledaren som har gjort ett bra jobb.
\newline

Vi har lärt oss hur man bättre kommunicerar med personer man inte känner från början för att tillsammans lösa problem.
\newline

\subsection{Projektmodell}

Vi använde projektmodellen i stora lag som man skulle med undantag av att vi hade väldigt otydliga iterationer i underfasen.

\subsection{Relation till beställare}

Relationen till beställaren har varit bra.

\subsection{Relation till handledaren}

Relationen till beställaren har varit bra.

\subsection{Tekniska framgångar/problem}

%saker som kunde ha varit bättre
\textbf{Kunde ha varit bättre:}
\newline
Vi borde ha funderat mer på vilka sensorer vi behövde.
\newline

Banreglerna borde vi ha lagt ner mer tid på och specat mer detaljerat. Oklara banregler ledde till en massa extra testning.
\newline

Vi hade problem med att läsa USART på AVR, vi fick till slut strunta i att göra det.
\newline

Motorns dokumentationslapp var fel.
\newline

%saker som var bra
\textbf{Bra:}
\newline
Beagleboard var en jättebra idé.
\newline

Linjesor i mitten var en bra idé.
\newline

Ordentliga levelshifters var en bra idé.