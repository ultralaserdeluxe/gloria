\documentclass[titlepage, a4paper]{article}
\usepackage[swedish]{babel}
\usepackage[utf8]{inputenc}
\usepackage{color}

% Sidformat
\usepackage{a4wide}

% Fixa Appendix-titlar
\usepackage[titletoc,title]{appendix}

% Bättre bildtexter
\usepackage[margin=10pt,font=small,labelfont=bf,labelsep=endash]{caption}

% Enkelt kommando som låter mig attgöra-markera text
\newcommand{\todo}[1] {\textbf{\textcolor{red}{#1}}}

%% Headers och Footers
\usepackage{fancyhdr}
\pagestyle{fancy}
\lhead{}
\rhead{\today}
\lfoot{\LIPSkursnamn \\ \LIPSdokumenttyp}
\cfoot{\thepage}
\rfoot{\LIPSprojektgrupp \\ \LIPSprojektnamn}

%% Titelsida
\newcommand{\LIPSTitelsida}{%
{\ }\vspace{45mm}
\begin{center}
  \textbf{\Huge \LIPSdokument}
\end{center}
\begin{center}
  {\Large Redaktör: \LIPSredaktor}
\end{center}
\begin{center}
  {\Large \textbf{Version \LIPSversion}}
\end{center}
\vfill
\begin{center}
  {\large Status}\\[1.5ex]
  \begin{tabular}{|*{3}{p{40mm}|}}
    \hline
    Granskad & \LIPSgranskare & \LIPSgranskatdatum \\
    \hline
    Godkänd & \LIPSgodkannare & \LIPSgodkantdatum \\
    \hline
  \end{tabular}
\end{center}
\newpage
}


% Projektidentitet
\newenvironment{LIPSprojektidentitet}{%
{\ }\vspace{45mm}
\begin{center}
  {\Large PROJEKTIDENTITET}\\[0.5ex]
  {\small
  \LIPSartaltermin, \LIPSprojektgrupp\\
  Linköpings Tekniska Högskola, MAI
  }
\end{center}
\begin{center}
  {\normalsize Gruppdeltagare}\\
  \begin{tabular}{|l|l|p{25mm}|l|}
    \hline
    \textbf{Namn} & \textbf{Ansvar} & \textbf{Telefon} & \textbf{E-post} \\
    \hline
}%
{%
    \hline
  \end{tabular}
\end{center}
\begin{center}
  {\small
    \textbf{E-postlista för hela gruppen}: \LIPSgruppadress\\
    \textbf{Hemsida}: \LIPSgrupphemsida\\[1ex]
    \textbf{Kund}: \LIPSkund\\
    \textbf{Kontaktperson hos kund}: \LIPSkundkontakt\\
    \textbf{Kursansvarig}: \LIPSkursansvarig\\
    \textbf{Handledare}: \LIPShandledare\\
  }
\end{center}
\newpage
}
\newcommand{\LIPSgruppmedlem}[4]{\hline {#1} & {#2} & {#3} & {#4} \\}

%% Dokumenthistorik
\newenvironment{LIPSdokumenthistorik}{%
\begin{center}
  Dokumenthistorik\\[1ex]
  %\begin{small}
    \begin{tabular}{|l|l|p{60mm}|l|l|}
      \hline
      \textbf{Version} & \textbf{Datum} & \textbf{Utförda förändringar} & \textbf{Utförda av} & \textbf{Granskad} \\
      }%
    {%
			\hline
    \end{tabular}
  %\end{small}
\end{center}
}

\newcommand{\LIPSversionsinfo}[5]{\hline {#1} & {#2} & {#3} & {#4} & {#5} \\}

% Kravlistor
\newenvironment{LIPSkravlista}{
	\center
		\tabularx{\textwidth}{| p{1.2cm} | p{1.9cm} | X | c |}
			\hline
			\textbf{Krav} & \textbf{Förändring} & \textbf{Beskrivning} & \textbf{Prioritet} \\\hline
}
{
		\endtabularx
	\endcenter
}

\newcounter{LIPSkravnummer}
\addtocounter{LIPSkravnummer}{1}
\newcommand{\LIPSkrav}[4][Krav \arabic{LIPSkravnummer}]{{#1} & {#2} & {#3} & {#4} \stepcounter{LIPSkravnummer}\\\hline}	% Importera generella layout-strukturer

\newcommand{\LIPSdokumenttyp}{Statusrapport}
\newcommand{\LIPSkursnamn}{TSEA29}
\newcommand{\LIPSprojektnamn}{Lagerrobot}
\newcommand{\LIPSprojektgrupp}{Grupp 2}
\newcommand{\LIPSutfardare}{palka285}	% Vem utfärdar statusrapporten?

\pagenumbering{roman}	% Sidnumrering

\begin{document}
{\ }\vspace{5mm}

\centerline{\textbf{\Huge Statusrapport}}
\vspace{2mm}
\centerline{\LARGE MALL} % MALL = Dagens datum YYYY-MM-DD
\vspace{5mm}

\begin{description}
  \item[Distributionslista:] Peter Johansson, Pål Kastman, Tomas Svensson.	% Vem skall delges statusrapporten
  \item[Nästa statusrapport:] YYYY-MM-DD	% När kommer nästa statusrapport?
  \item[Beräknad färdigtidpunkt:] 2014-12-19	% När blir projektet klart?
\end{description}

\vspace{7mm}
\textbf{\Large Pågående aktiviteter}

\begin{statusrapport}
% ------------ PÅGÅENDE AKTIITETER -------------
% Syntax \aktivitetstatus{Nr}{Aktivitet}{Kommentar}{Planerad tid}{Nedlagd tid}{Planerad klar}{Beräknad klar}
  \aktivitetstatus{1}{Implementation av protokoll mellan huvud- och styrenheten}{}{16}{}{2014-11-07}{}
  \aktivitetstatus{2}{Koppla ihop huvud- och styrenheten}{}{4}{}{2014-11-07}{}
  \aktivitetstatus{3}{Implementera buss mellan huvud- och sensorenheten}{}{16}{}{2014-11-07}{}
  \aktivitetstatus{4}{Koppla ihop huvud- och sensorenheten}{}{4}{}{2014-11-07}{}
  \aktivitetstatus{5}{Sätta upp utvecklingsmiljö för AVR}{}{8}{}{2014-11-07}{}
  \aktivitetstatus{6}{Installera mjukvara på huvud (OS, Python, drivare)}{}{10}{}{2014-11-07}{}
  \aktivitetstatus{7}{Python-modul på PC för att skicka/ta emot data från huvudenheten}{}{16}{}{2014-11-14}{}
  \aktivitetstatus{8}{Koppla in linjesensorer på sensorenheten}{}{10}{}{2014-11-07}{}
\end{statusrapport}	% Använd för att skapa sidbrytning i tabellen
\begin{statusrapport}
  \aktivitetstatus{9}{Koppla in IR-sensor på sensorenheten}{}{2}{}{2014-11-07}{}
  \aktivitetstatus{10}{Koppla in motorer och styrenheten}{}{2}{}{2014-11-07}{}
  \aktivitetstatus{11}{Koppla in servon på styrenheten}{}{2}{}{2014-11-07}{}
  \aktivitetstatus{12}{Upprätta BT förbindelse mellan huvudenheten och PC}{6}{10}{}{2014-11-07}{}
  \aktivitetstatus{13}{Sätta upp utvecklingsmiljö för BeagleBoard (wifi)}{6}{4}{}{2014-11-07}{}
% ----------------------------------------------
\end{statusrapport}
	
\vspace{7mm}
\textbf{\Large Avslutade aktiviteter}

\begin{statusrapport}
% ------------ AVSLUTADE AKTIITETER ------------
	\aktivitetstatus{55}{Dokumentation: Designspecifikation}{}{100}{104}{2014-10-07}{2014-10-03}
% ----------------------------------------------
\end{statusrapport}

\vspace{7mm}
\textbf{\Large Nära förstående aktiviteter}

\begin{statusrapport}
% ------------ NÄRA FÖRSTÅENDE AKTIITETER ------
  \aktivitetstatus{14}{Skriva UI för PC}{7}{20}{}{2014-11-14}{}
  \aktivitetstatus{15}{Implementera och testa muxning för linjesensor}{8}{8}{}{2014-11-14}{}
  \aktivitetstatus{16}{Implementera styrning av motorer}{10}{8}{}{2014-11-14}{}
  \aktivitetstatus{17}{Implementera styrning av servon}{11}{20}{}{2014-11-14}{}
  \aktivitetstatus{18}{Testa och felsöka buss mellan huvud- och styrenheten}{1,2,13}{16}{}{2014-11-14}{}
  \aktivitetstatus{19}{Testa och felsöka buss mellan huvud- och sensorenheten}{3,4,13}{16}{}{2014-11-14}{}
\end{statusrapport}	% Använd för att skapa sidbrytning i tabellen
\begin{statusrapport}
  \aktivitetstatus{24}{Implementera styrlogik}{13}{40}{}{2014-12-12}{}
  \aktivitetstatus{25}{Implementera läsning av sensorer}{15,9}{16}{}{2014-11-21}{}
  \aktivitetstatus{26}{Mäta respons från motorer och servon}{16,17}{8}{}{2014-11-21}{}
  \aktivitetstatus{27}{Implementera tolkning och utförande av kommandon från PC på huvudenheten}{14,19}{12}{}{2014-11-21}{}
  \aktivitetstatus{28}{Tolka IR-sensordata}{25}{8}{}{2014-11-21}{}
  \aktivitetstatus{29}{Implementera kalibreringsfunktion}{37}{20}{}{2014-11-28}{}
  \aktivitetstatus{30}{Implementera tolkning och utförande av kommandon från huvudenheten på sensorenheten}{19,25}{16}{}{2014-11-21}{}
  \aktivitetstatus{31}{Testa styra servon från styrenheten}{26}{8}{}{2014-11-14}{}
  \aktivitetstatus{32}{Testa styra motorer från styrenheten}{26}{8}{}{2014-11-14}{}
  \aktivitetstatus{33}{Testa styra servon från huvudenheten}{18,31}{8}{}{2014-11-21}{}
  \aktivitetstatus{34}{Testa styra motorer från huvudenheten}{18,32}{8}{}{2014-11-21}{}
  \aktivitetstatus{35}{Implementera läsning av sensordata på huvudenheten}{19,30}{8}{}{2014-11-21}{}
  \aktivitetstatus{37}{Skriva testprogram för linjesensor styrenhet}{25}{8}{}{2014-11-21}{}
  \aktivitetstatus{38}{Implementera och testa X,Y,Z till servovinkel, konvertering}{31}{32}{}{2014-12-05}{}
  \aktivitetstatus{39}{Implementera och testa gränser för armen}{31}{24}{}{2014-11-28}{}
\end{statusrapport}	% Använd för att skapa sidbrytning i tabellen
\begin{statusrapport}
  \aktivitetstatus{40}{Implementera smoothing-funktion för servon och motorer}{32}{16}{}{2014-12-12}{}
  \aktivitetstatus{41}{Implementera paketnersättningsfunktion}{33}{16}{}{2014-11-28}{}
  \aktivitetstatus{42}{Implementera fjärrstyrning från PC}{27,33,34,35}{16}{}{2014-12-05}{}
  \aktivitetstatus{43}{Implementera och testa detektion av stoppmarkering}{35,37}{16}{}{2014-12-05}{}
  \aktivitetstatus{44}{Implementera och testa detektion av paket}{35,36}{16}{}{2014-12-05}{}
  \aktivitetstatus{45}{Implementera och testa detektion av stationer}{35,37}{16}{}{2014-12-05}{}
  \aktivitetstatus{46}{Implementera regleringsalgoritm (linjeföljare)}{34,37}{40}{}{2014-11-28}{}
  \aktivitetstatus{47}{Testa styrlogik}{43,44,45,46}{40}{}{2014-12-19}{}
  \aktivitetstatus{48}{Möten}{}{72}{}{2014-12-19}{}
  \aktivitetstatus{49}{Dokumentation: Teknisk Dokumentation}{}{32}{}{2014-12-14}{}
  \aktivitetstatus{50}{Dokumentation: Tidsrapport}{}{10}{}{2014-12-15}{}
  \aktivitetstatus{51}{Dokumentation: Användarhandledning}{}{8}{}{2014-12-14}{}
  \aktivitetstatus{52}{Dokumentation: Efterstudie}{}{8}{}{2014-12-14}{}
  \aktivitetstatus{53}{Presentation + PP}{}{32}{}{2014-12-16}{}
  \aktivitetstatus{54}{Tejpa testbanor}{}{3}{}{2014-11-21}{}
  \aktivitetstatus{56}{Buffertid}{}{100}{}{}{}
% ----------------------------------------------
\end{statusrapport}

\vspace{7mm}

\newpage
\textbf{\Large Förbrukade resurser}

% ------------ FÖRBRUKADE RESURSER -------------
% Löpande text om hur mycket vi förbrukat
% ----------------------------------------------

\vspace{7mm}
\textbf{\Large Problem och risker}

\begin{center}
\begin{tabularx}{\textwidth}{| X | X | p{13.5mm} |}
	\hline
	\textbf{Beskrivning} & \textbf{Åtgärdsförslag} & \textbf{Ansvar} \\\hline	
% ------------ PROBLEM OCH RISKER --------------
% Exempel Beskrivning & Åtgärdsförslag & Ansvar \\\hline
% ----------------------------------------------
\end{tabularx}
\end{center}

\vspace{7mm}
\textbf{\Large Möjligheter}
\begin{center}
\begin{tabularx}{\textwidth}{| X | X | p{13.5mm} |}
	\hline
	\textbf{Beskrivning} & \textbf{Åtgärdsförslag} & \textbf{Ansvar} \\\hline	
% ------------ MÖJLIGHETER ---------------------
% ----------------------------------------------
\end{tabularx}
\end{center}

\end{document}
