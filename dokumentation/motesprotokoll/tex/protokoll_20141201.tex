\documentclass[titlepage, a4paper]{article}
\usepackage[swedish]{babel}
\usepackage[utf8]{inputenc}
\usepackage{color}
\usepackage{graphicx}
\usepackage{etoolbox}

% Sidformat
\usepackage{a4wide}

% Fixa Appendix-titlar
\usepackage[titletoc,title]{appendix}

% Bättre bildtexter
\usepackage[margin=10pt,font=small,labelfont=bf,labelsep=endash]{caption}

% Enkelt kommando som låter mig attgöra-markera text
\newcommand{\todo}[1] {\textbf{\textcolor{red}{#1}}}

%% Headers och Footers
\usepackage{fancyhdr}
\pagestyle{fancy}
\lhead{\includegraphics[scale=0.12]{../logga/Logga1.png}}
\rhead{\today}
\lfoot{\LIPSkursnamn \\ \LIPSdokumenttyp}
\cfoot{\thepage}
\rfoot{\LIPSprojektgrupp \\ \LIPSprojektnamn}

%% Titelsida
\newcommand{\LIPSTitelsida}{%
{\ }\vspace{45mm}
\begin{center}
  \textbf{\Huge \LIPSdokument}
\end{center}
\begin{center}
  {\Large Redaktör: \LIPSredaktor}
\end{center}
\begin{center}
  {\Large \textbf{Version \LIPSversion}}
\end{center}
\vfill
\begin{center}
  {\large Status}\\[1.5ex]
  \begin{tabular}{|*{3}{p{40mm}|}}
    \hline
    Granskad & \LIPSgranskare & \LIPSgranskatdatum \\
    \hline
    Godkänd & \LIPSgodkannare & \LIPSgodkantdatum \\
    \hline
  \end{tabular}
\end{center}
\newpage
}


% Projektidentitet
\newenvironment{LIPSprojektidentitet}{%
{\ }\vspace{45mm}
\begin{center}
  {\Large PROJEKTIDENTITET}\\[0.5ex]
  {\small
  \LIPSartaltermin, \LIPSprojektgrupp\\
  Linköpings Tekniska Högskola, ISY
  }
\end{center}
\begin{center}
  {\normalsize Gruppdeltagare}\\
  \begin{tabular}{|l|l|p{25mm}|l|}
    \hline
    \textbf{Namn} & \textbf{Ansvar} & \textbf{Telefon} & \textbf{E-post} \\
    \hline
}%
{%
    \hline
  \end{tabular}
\end{center}
\begin{center}
  {\small
    \ifdef{\LIPSgruppadress}{\textbf{E-postlista för hela gruppen}: \LIPSgruppadress\\}{}
    \ifdef{\LIPSgrupphemsida}{\textbf{Hemsida}: \LIPSgrupphemsida\\[1ex]}{}
    \ifdef{\LIPSkund}{\textbf{Kund}: \LIPSkund\\}{}
    \ifdef{\LIPSkundkontakt}{\textbf{Kontaktperson hos kund}: \LIPSkundkontakt\\}{}
    \ifdef{\LIPSkursansvarig}{\textbf{Kursansvarig}: \LIPSkursansvarig\\}{}
    \ifdef{\LIPShandledare}{\textbf{Handledare}: \LIPShandledare\\}{}
  }
\end{center}
\newpage
}
\newcommand{\LIPSgruppmedlem}[4]{\hline {#1} & {#2} & {#3} & {#4} \\}

%% Dokumenthistorik
\newenvironment{LIPSdokumenthistorik}{%
\begin{center}
  Dokumenthistorik\\[1ex]
  %\begin{small}
    \begin{tabular}{|l|l|p{60mm}|l|l|}
      \hline
      \textbf{Version} & \textbf{Datum} & \textbf{Utförda förändringar} & \textbf{Utförda av} & \textbf{Granskad} \\
      }%
    {%
			\hline
    \end{tabular}
  %\end{small}
\end{center}
}

\newcommand{\LIPSversionsinfo}[5]{\hline {#1} & {#2} & {#3} & {#4} & {#5} \\}

% Kravlistor
\newenvironment{LIPSkravlista}{
	\center
		\tabularx{\textwidth}{| p{1.2cm} | p{1.9cm} | X | c |}
			\hline
			\textbf{Krav} & \textbf{Förändring} & \textbf{Beskrivning} & \textbf{Prioritet} \\\hline
}
{
		\endtabularx
	\endcenter
}

\newcounter{LIPSkravnummer}
\addtocounter{LIPSkravnummer}{1}
\newcommand{\LIPSkrav}[4][Krav \arabic{LIPSkravnummer}]{{#1} & {#2} & {#3} & {#4} \stepcounter{LIPSkravnummer}\\\hline}	% Importera generella layout-strukturer
\usepackage{tabularx}

\newcommand{\LIPSdokumenttyp}{Mötesprotokoll}
\newcommand{\LIPSkursnamn}{TSEA29}
\newcommand{\LIPSprojektnamn}{Lagerrobot}
\newcommand{\LIPSprojektgrupp}{Grupp 2}
\newcommand{\LIPSutfardare}{hansn314} % Vem har utfärdat protokollet?

\pagenumbering{roman}	% Sidnumrering

\begin{document}

\begin{projektmote}{Mötesprotokoll}{2014-12-01}
% ------------ PARAGRAFER ----------------------
% Syntax \paragraf{Vi bestämde och kom överens}
% ----------------------------------------------
\paragraf{Mötet öppnas.}
\paragraf{Närvarande: Martin, Daniel, Dennis, Pål, Yngve och Hannes}
\paragraf{Vi går igenom bordlagda frågor angående statusrapporten:
\begin{itemize}
\item{\textbf{Hur mycket tid har ni kvar av era budgeterade timmar?} \newline Vi har 354 budgeterade timmar kvar att lägga på projektet.}
\item{\textbf{Hur många timmar har respektive projektmedlem kvar att leverera och hur ska dessa timmar fördelas över de kvarvarande veckorna? Redovisa i en tabell i statusrapporten hur många timmar detta blir per person och vecka. Redovisa också vilka aktiviteter som respektive person ska arbeta med.} \newline 
}
\end{itemize}
}
\begin{table}[h]
\center
\begin{tabularx}{\textwidth}{| X | X | l | l | l | X |}
	\hline
	{\textbf{Vem}} & {\textbf{Sammanlagt}} & {\textbf{v.49}} & {\textbf{v.50}} & {\textbf{v.51}} & {\textbf{Aktiviteter}} \\\hline
	{Pål} & {63} & {25} & {25} & {13} & {48,41,44,53,50,42} \\\hline
	{Hannes} & {46} & {18} & {18} & {10} & {48,49,26,45} \\\hline
	{Martin} & {55} & {20} & {20} & {15} & {48,38,39,46,53} \\\hline
	{Dennis} & {72} & {27} & {27} & {18} & {48,24,40,44,47,53} \\\hline
	{Yngve} & {53} & {20} & {20} & {13} & {48,46,43,49}  \\\hline
	{Daniel} & {67} & {25} & {25} & {17} & {48,47,42,51,52}  \\\hline

\end{tabularx}
\end{table}
\paragraf{Mötet avslutas.}
\end{projektmote}

\newpage
\textbf{\Large Aktuella aktiviteter}
\begin{center}
\begin{tabularx}{\textwidth}{| p{4mm} | X | p{13.5mm} | X |}
	\hline
	\textbf{Nr} & \textbf{Aktivitet} & \textbf{Ansvar} & \textbf{Status} \\\hline	
% ------------ Aktuella aktiviteter ------------
% Syntax {Nr} & {Aktivitet} & {Ansvarig} & {Status} \\\hline
% ----------------------------------------------
	{14} & {Skriva UI för PC} & {DW} & {En första iteration färdig} \\\hline
	{17} & {Implementera styrning av servon} & {HS} & {Klarr} \\\hline
	{24} & {Implementera styrlogik} & {DL} & {Påbörjad} \\\hline
	{26} & {Mäta respons från motorer och servon} & {AY} & {Ej påbörjad, Försenad} \\\hline
	{28} & {Tolka IR-sensordata} & {PK} & {Påbörjad} \\\hline
	{33} & {Testa styra servon från huvud} & {AY} & {Klar} \\\hline
	{35} & {Implementera läsning av sensordata på huvud} & {DL} & {Påbörjad} \\\hline
	{38} & {Implementera och testa X,Y,Z  till servovinkel konvertering} & {MS, HS} & {Ej testat, annars klart} \\\hline
	{39} & {Implementera och testa gränser för armen} & {MS} & {Ej påbörjad, försenad} \\\hline
	{40} & {Implementera smoothing-funktion för servon och motorer} & {DL} & {Ej påbörjad} \\\hline
	{41} & {Implementera paketnedsättningsfunktion} & {PK} & {Ej påbörjad, försenad} \\\hline
	{42} & {Implementera fjärrstyrning från PC} & {DW} & {Påbörjad} \\\hline
	{43} & {Implementera detektion av stoppmarkering} & {} & {påbörjad} \\\hline
	{44} & {Implementera detektion av paket} & {} & {Påbörjad} \\\hline
	{45} & {Implementera detektion av stationer} & {} & {Påbörjad} \\\hline
	{46} & {Implementera regleringsalgoritm} & {AY, MS} & {Implementerad, behöver finjusteras} \\\hline
	{54} & {Tejpa testbanor} & {DW} & {Ej påbörjad} \\\hline
\end{tabularx}
\end{center}

\end{document}