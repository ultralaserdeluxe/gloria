\section{Resursplan}

\subsection{Personer}

\subsection{Material}
\begin{table}[h]
	\centering
		\begin{tabularx}{\textwidth}{| X | l |}
			\hline
			\textbf{Material} & \textbf{Tillgänglighet} \\
			\hline
			Robotplattform &  \\
			\hline
			Robotarm &  \\
			\hline
			BeagleBoard &  \\
			\hline
			2 AVR av typ ATmega16 &  \\
			\hline
			2 linjesensorer &  \\
			\hline
			2 IR-sensorer &  \\
			\hline
			AVRStudio &  \\
			\hline
			JTAG ICE mkII & \\
			\hline
			Logikanalysator & \\
			\hline
			Virkort & \\
			\hline

		\end{tabularx}
	\caption{Nödvändig materiel} \label{}
\end{table}

All material nödvändig för projektet finns tillgänglig.


\subsection{Lokaler}
Projektgruppen kommer ha tillgång till Muxen på campus Valla för att konstruera hårdvaran för roboten. Där kommer gruppen ha tillgång till en eller två arbetsplatser. Arbetet kommer försökas delas upp på ett sådant sätt att inte alla gruppmedlemmar behöver vistas vid arbetsstationerna samtidigt. Viss utveckling av framförallt mjukvara kommer vara möjligt att utföra enskilt hemifrån. \\
Möten kommer hållas antingen i det konferensrum som finns tillgängligt i Muxen eller i annan lokal som finns tillgäng på universitetet.

\subsection{Ekonomi}

Projektet har en övre tidsgräns på 960 timmar. Dessa redovisas veckovis till beställaren.