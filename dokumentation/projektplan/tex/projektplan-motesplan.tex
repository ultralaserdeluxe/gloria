\section{Mötesplan}

Projektledaren sammankallar till projektmöte. Målet är att ha två möten i veckan. Med det första mötet för veckan avses att stämma av hur gruppen ligger till och om några oförutsedda problem har uppstått. Detta möte förväntas inte ta längre tid än en halv timme. I slutet av varje vecka hålls ett möte för att utvärdera och sammanställa veckans arbetsinsatser samt planera efterföljande veckas arbete. Beslut om nödvändiga förändringar i tidsplanen tas av projektledaren i slutet av varje vecka. Detta möte beräknas därför ta mer tid. \\
Projektmöten hålls efter en av två mötesmallar (Se Bilaga \ref{ap:motesmall}) beroende på vilken typ av möte det är. Mötestyp 1 är menat för kortare möte mitt i veckan för att fånga upp problem så tidigt som möjligt. Mötestyp 2 är menat för något längre möte i slutet av arbetsveckan för att summera den gångna veckan och planera efterföljande arbetsvecka.
