\section{Resursplan}

\subsection{Personer}
Projektgruppen består av medlemmar enligt tabell \ref{projektplan:resursplan-personer}
\begin{table}[h]
	\centering
		\begin{tabularx}{\textwidth}{| l | l | X | l | l | l |}
			\hline
			\textbf{Namn} & \textbf{Ansvar} & \textbf{E-post} \\
			\hline
			{Pål Kastman} & {Projektledare} & {palka285@student.liu.se} \\\hline
			{Daniel Wassing} & {Leveransansvarig} & {danwa223@student.liu.se} \\\hline
			{Hannes Snögren} & {Dokumentansvarig} & {hansn314@student.liu.se} \\\hline
			{Martin Söderén} & {Mjukvaruansvarig} & {marso329@student.liu.se} \\\hline
			{Alexander Yngve} & {Hårdvaruansvarig} & {aleyn573@student.liu.se} \\\hline
			{Dennis Ljung} & {Testansvarig} & {denlj069@student.liu.se} \\\hline
		\end{tabularx}
	\caption{Medlemmar i projektgruppen} \label{projektplan:resursplan-personer}
\end{table}

\subsection{Material}
Material nödvändig för projektet kommer att förses av beställaren. Om något saknas under projektets gång kontaktar projektledaren beställaren för att undersöka möjligheter att införskaffa detta.

\subsection{Lokaler}
Projektgruppen kommer ha tillgång till Muxen på campus Valla för att konstruera hårdvaran för roboten. Där kommer gruppen ha tillgång till en eller två arbetsplatser. Arbetet kommer försökas delas upp på ett sådant sätt att inte alla gruppmedlemmar behöver vistas vid arbetsstationerna samtidigt. Viss utveckling av framförallt mjukvara kommer vara möjligt att utföra enskilt hemifrån. \\
Möten kommer hållas antingen i det konferensrum som finns tillgängligt i Muxen eller i annan lokal som finns tillgäng på universitetet.

\subsection{Ekonomi}
Projektet har en övre tidsgräns på 960 arbetstimmar efter beslutspunkt 2. Dessa redovisas veckovis till beställaren.