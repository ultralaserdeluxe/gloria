\section{Milstolpar och beslutspunkter}
Beslutspunkter är uppsatta enligt LIPS-standarden. Milstolpar är organiserade så att kommunikation mellan moduler skall avklaras först, därefter läggs funktionalitet för roboten på allt eftersom.

\subsection{Milstolpar}
Nedan följer milstolpar uppsatta för projektet.
\begin{LIPSmilstolpar}
\LIPSmilstolpe{1}{Fungerande kommunikation mellan huvud-, styr- och sensorenhet}{2014-11-14}
\LIPSmilstolpe{2}{Läs data (driftinfo, sensorvärden) på PC}{2014-11-21}
\LIPSmilstolpe{3}{Robotens samtliga motorer och servon kan styras från PC}{2014-11-21}
\LIPSmilstolpe{4}{Roboten kan följa en linje autonomt}{2014-11-28}
\LIPSmilstolpe{5}{Robotens arm har full funktionalitet}{2014-12-05}
\LIPSmilstolpe{6}{Roboten kan stanna på plockstationer och detektera paket}{2014-12-12}
\LIPSmilstolpe{7}{Roboten är tävlingsklar}{2014-12-19}
\end{LIPSmilstolpar}

\subsection{Beslutspunkter}
\begin{LIPSmilstolpar}
\LIPSmilstolpe{0}{Godkännande av projektdirektiv, beslut att starta förstudie}{2014-09-04}
\LIPSmilstolpe{1}{Godkännande av kravspecifikation, beslut att starta förberedelsefasen}{2014-09-16}
\LIPSmilstolpe{2}{Godkännande av projektplanering, beslut att starta utförandefasen}{2014-10-02}
\LIPSmilstolpe{3}{Godkännande av designspecifikationen, beslut att fortsätta utförandefasen}{2014-11-07}
\LIPSmilstolpe{4}{Används ej}{}
\LIPSmilstolpe{5}{Godkännande av produktens funktionalitet, beslut att leverera}{2014-12-18}
\LIPSmilstolpe{6}{Godkännande av leverans, beslut att upplösa projektgruppen}{2014-12-19}
\end{LIPSmilstolpar}
