\section{Konfigurering av Gloria}

\subsection{Ansluta till Gloria}
För att kunna ansluta till huvudenheten behöver följande göras:
\begin{enumerate}
	\item Ha tillgång till en terminal på huvudenheten, det enklaste är att ansluta en skärm och tangentbord till den.\newline Lösenordet är \textbf{temppwd}
	\item Kör \textbf{sudo bluez-simple-agent hci0 xx:xx:xx:xx:xx:xx} där du byter ut kryssen mot din blåtandenhets mac-adress.
	\item \textbf{Paira} (förbind) huvudenheten med din pc från din pc. Huvudenheten är listad som arm-0 via blåtand.
	\item Kör \textbf{sudo bluez-test-device trusted xx:xx:xx:xx:xx:xx yes} på huvudenheten där du återigen byter ut kryssen mot din blåtandenhets mac-adress.
	\item Du kan nu ta bort pairing (men låt blåtand fortsatt vara påslaget) från din pc.
	\item Kör \textbf{sudo pand -n -c 00:19:0E:0F:F0:6F	} på din pc.
	\item Kör \textbf{sudo ifconfig bnep0 192.168.99.2 up } på din pc.
\end{enumerate}
Nu borde du kunna pinga och ssh in på ubuntu@192.168.99.1 vilket är huvudenhetens statiska ip-adress. I framtiden behövs bara de två sista stegen genmföras.
\subsection{Starta Gloria}
\begin{enumerate}
	\item Om anslutning har etablerats till Gloria så behöver man hitta filen mainThread.py som ligger i home directory, dvs där man hamnar när man väl har ssh:at in (se Användning av Systemet).
	\item Kommandot för att starta systemet i Gloria är bara en rad: \textbf{python mainThread.py}
	När detta kommandot körs så sätter systemet igång och Gloria lägger sig i vänteläge.
	\item Efter att mainThread.py har startats på Gloria så startar man upp en ny terminal lokalt på datorn som Gloria ska styras ifrån och hittar Gui.py. Sedan kör man kommandot: \textbf{python Gui.py}
	\item Anslut två joysticks via USB (till pc). Klargör vilken som styr motor och vilken som styr arm.
\end{enumerate}
