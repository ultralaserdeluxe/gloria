\section{Användning av systemet}

\subsection{Ansluta till Huvudenheten}
\todo{Direkt från tekdok. Kan behöva redigeras för att passa in}
För att ansluta en ny dator med Linux till huvudenhetens PAN behövs följande göras:
\todo{får uppdatera detta, osäker om detta stämmer}
\begin{enumerate}
	\item Ha tillgång till en terminal på huvudenheten, det enklaste är att ansluta en skärm och tangentbord till den.\newline Lösenordet är \textbf{temppwd}
	\item Kör \textbf{sudo bluez-simple-agent hci0 xx:xx:xx:xx:xx:xx} där du byter ut kryssen mot din blåtandenhets mac-adress
	\item Paira huvudenheten med din pc från din pc
	\item Kör \textbf{sudo bluez-test-device trusted xx:xx:xx:xx:xx:xx yes} på huvudenheten
	\item Du kan nu ta bort pairing från din pc
	\item Kör \textbf{sudo pand -n -c 00:19:0E:0F:F0:6F	} på din pc
	\item Kör \textbf{sudo ifconfig bnep0 192.168.99.2 up } på din pc
\end{enumerate}
Nu borde du kunna pinga och ssh in på ubuntu@192.168.99.1 vilket är huvudenhetens statiska ip-adress. I framtiden behövs bara de två sista stegen genmföras.
