\section{Styrning av Gloria}

\subsection{Användning av det grafiska gränssnittet}
\begin{enumerate}
	\item Styrning av Gloria sker som följer: 
	\begin{itemize}
		\item Väl inne i Gui.py, tryck på \textbf{connect arm joystick} för att ansluta joysticken som styr armen.
		\item Samma procedur gäller för joysticken som styr motorn.
		\item Ange Glorias statiska IP-adress (192.168.99.1) i connect-fältet och tryck på \textbf{connect}.
		\item Motor: Joystick 1 låter användaren köra i alla riktningar. Framåt-tilt får Gloria att accelerera framåt, bakåt-tilt ger samma effekt bakåt. Vänster orsakar en vänstersväng (vänster hjulpar kör långsammare än höger) och höger fungerar på samma sätt.
		\item Arm: Armen (Gripklon) styrs i ett 3D-rum utifrån hur man rör Joystick 2 där gripklon är koordinaten i planet. Rörelserna beräknas automatiskt, så dessa behöver användaren inte ta hänsyn till. 
		\item Gripklo framåt/bakåt/vänster/höger: Tilta Joysticken i önskad rikting.
		\item Höj/Sänk Gripklon: Vrid Joysticken åt vänster/höger.
		\item Rotera vristen: Högra spaken bak tiltas från 0 till 1.
		\item Gripklo (släpp, grepp): Vänster knapp på framsidan av Joysticken regleras från T1 till T2, där T1 innebär att gripklon öppnar sig och T2 att den stänger sig.
	\end{itemize}
	\item Knapparna i nedre vänstra hörnet på GUI gör följande:
	\begin{itemize}
		\item start: Startar Glorias mainloop, behöver göras innan några andra kommandon kan accepteras.
		\item calibrate tape: Kalibrerar sensorvärdet för tejp.
		\item calibrate floor: Kalibrerar sensorvärdet för golv.
		\item got package: Signalerar till Gloria att man är klar med att plocka upp ett packet med Joysticken.
		\item auto motor: Sätter motorerna i autonomt läge, vilket får Gloria att köra av sig själv givet sensordata.
		\item auto arm: Sätter armen i autonomt läge.
	\end{itemize}
	\item För att garantera att Gloria ska fungera enligt specifikationerna så krävs det att banreglerna som visas i Appendix A följs.
	\begin{figure}[h!]
	\center
	\includegraphics[scale=0.6]{Gui.png}
	\endcenter
	\caption{Gui. Överst visas sensordata, i mitten visas gloria, till vänster alla relevanta knappar, till höger visas reglerfel samt errorcodes och nederst visas en terminal.}
	\end{figure}
	\newpage
\end{enumerate}