\section{Teori}
I projektet ingår främst två större teoretiska beräkningsmoment. Vi måste reglera utdatan till motorerna för att roboten skall följa linjen. Vidare vill vi kunna använda enklare koordinater för att tala om vart i rummet armen befinner sig, och vi behöver således kunna konvertera $(x,y,z)$-koordinater till vinklar för armens axlar.

\subsection{Reglering}
\todo{Hur fungerar regleringen!?}

\subsection{Styrning av arm}
Omvandlingen från $(x,y,z)$-koordinater till vinklar för armens servon sker på huvudenheten. På huvudenheten finns en pythonmodul \texttt{arm.py} som implementerar beräkningarna som beskrivs nedan. Den används i tre steg.
\begin{enumerate}
	\item Skapa en instans av \texttt{robotArm} som är en klass i \texttt{arm.py}.
	\item Använd funktionen \texttt{setAll} som tar in en lista som argument med följande innehåll \newline \texttt{[x,y,z,gripperAngle,gripperRotationsOffset,gripper]}.
	\item Använd funktionen \texttt{getServoValues} som returnerar en lista med alla servons vinklar i form av 10-bitars heltal.
\end{enumerate}

GripperAngle är gripklons vinkel mot marken, gripperRotationsOffset är en offset på gripklons vinkel i förhållande till leden som gripklons är fäst i och gripper är hur mycket gripklon klämmer.

I listan som returneras är första värdet till armens bas, det andra är till $A1$, det tredje är till $A2$, det fjärde till $A3$, det femte till klons rotation och det sjätte till klons grepp.

\begin{figure}[h!]
	\centerline{\input{grafik/huvud-armkoordinatsystem}}
	\caption{Armens koordinatsystem i förhållande till roboten. Upp är robotens färdriktning}
\end{figure}

\begin{figure}[h!]
	\centerline{% Graphic for TeX using PGF
% Title: /home/martin/Diagram1.dia
% Creator: Dia v0.97.2
% CreationDate: Sun Oct 12 20:23:47 2014
% For: martin
% \usepackage{tikz}
% The following commands are not supported in PSTricks at present
% We define them conditionally, so when they are implemented,
% this pgf file will use them.
\ifx\du\undefined
  \newlength{\du}
\fi
\setlength{\du}{15\unitlength}
\begin{tikzpicture}
\pgftransformxscale{1.000000}
\pgftransformyscale{-1.000000}
\definecolor{dialinecolor}{rgb}{0.000000, 0.000000, 0.000000}
\pgfsetstrokecolor{dialinecolor}
\definecolor{dialinecolor}{rgb}{1.000000, 1.000000, 1.000000}
\pgfsetfillcolor{dialinecolor}
\pgfsetlinewidth{0.100000\du}
\pgfsetdash{}{0pt}
\pgfsetdash{}{0pt}
\pgfsetbuttcap
{
\definecolor{dialinecolor}{rgb}{0.000000, 0.000000, 0.000000}
\pgfsetfillcolor{dialinecolor}
% was here!!!
\definecolor{dialinecolor}{rgb}{0.000000, 0.000000, 0.000000}
\pgfsetstrokecolor{dialinecolor}
\draw (3.650000\du,5.762500\du)--(7.900000\du,-1.587500\du);
}
\pgfsetlinewidth{0.100000\du}
\pgfsetdash{}{0pt}
\pgfsetdash{}{0pt}
\pgfsetbuttcap
{
\definecolor{dialinecolor}{rgb}{0.000000, 0.000000, 0.000000}
\pgfsetfillcolor{dialinecolor}
% was here!!!
\definecolor{dialinecolor}{rgb}{0.000000, 0.000000, 0.000000}
\pgfsetstrokecolor{dialinecolor}
\draw (7.850000\du,-1.537500\du)--(17.900000\du,-1.487500\du);
}
\pgfsetlinewidth{0.100000\du}
\pgfsetdash{}{0pt}
\pgfsetdash{}{0pt}
\pgfsetbuttcap
{
\definecolor{dialinecolor}{rgb}{0.000000, 0.000000, 0.000000}
\pgfsetfillcolor{dialinecolor}
% was here!!!
\definecolor{dialinecolor}{rgb}{0.000000, 0.000000, 0.000000}
\pgfsetstrokecolor{dialinecolor}
\draw (17.850000\du,-1.437500\du)--(20.900000\du,2.962500\du);
}
\pgfsetlinewidth{0.100000\du}
\pgfsetdash{{1.000000\du}{1.000000\du}}{0\du}
\pgfsetdash{{1.000000\du}{1.000000\du}}{0\du}
\pgfsetbuttcap
{
\definecolor{dialinecolor}{rgb}{0.000000, 0.000000, 0.000000}
\pgfsetfillcolor{dialinecolor}
% was here!!!
\definecolor{dialinecolor}{rgb}{0.000000, 0.000000, 0.000000}
\pgfsetstrokecolor{dialinecolor}
\draw (3.600000\du,6.162500\du)--(3.600000\du,-7.087500\du);
}
\pgfsetlinewidth{0.100000\du}
\pgfsetdash{{1.000000\du}{1.000000\du}}{0\du}
\pgfsetdash{{1.000000\du}{1.000000\du}}{0\du}
\pgfsetbuttcap
{
\definecolor{dialinecolor}{rgb}{0.000000, 0.000000, 0.000000}
\pgfsetfillcolor{dialinecolor}
% was here!!!
\definecolor{dialinecolor}{rgb}{0.000000, 0.000000, 0.000000}
\pgfsetstrokecolor{dialinecolor}
\draw (3.700000\du,6.212500\du)--(27.500000\du,6.062500\du);
}
\pgfsetlinewidth{0.100000\du}
\pgfsetdash{}{0pt}
\pgfsetdash{}{0pt}
\pgfsetbuttcap
{
\definecolor{dialinecolor}{rgb}{0.000000, 0.000000, 0.000000}
\pgfsetfillcolor{dialinecolor}
% was here!!!
\definecolor{dialinecolor}{rgb}{0.000000, 0.000000, 0.000000}
\pgfsetstrokecolor{dialinecolor}
\pgfpathmoveto{\pgfpoint{6.049937\du}{6.212702\du}}
\pgfpatharc{18}{-65}{2.110604\du and 2.110604\du}
\pgfusepath{stroke}
}
\pgfsetlinewidth{0.100000\du}
\pgfsetdash{}{0pt}
\pgfsetdash{}{0pt}
\pgfsetbuttcap
{
\definecolor{dialinecolor}{rgb}{0.000000, 0.000000, 0.000000}
\pgfsetfillcolor{dialinecolor}
% was here!!!
\definecolor{dialinecolor}{rgb}{0.000000, 0.000000, 0.000000}
\pgfsetstrokecolor{dialinecolor}
\pgfpathmoveto{\pgfpoint{6.949854\du}{0.062444\du}}
\pgfpatharc{112}{9}{2.148827\du and 2.148827\du}
\pgfusepath{stroke}
}
\pgfsetlinewidth{0.100000\du}
\pgfsetdash{}{0pt}
\pgfsetdash{}{0pt}
\pgfsetbuttcap
{
\definecolor{dialinecolor}{rgb}{0.000000, 0.000000, 0.000000}
\pgfsetfillcolor{dialinecolor}
% was here!!!
\definecolor{dialinecolor}{rgb}{0.000000, 0.000000, 0.000000}
\pgfsetstrokecolor{dialinecolor}
\pgfpathmoveto{\pgfpoint{16.550022\du}{-1.537549\du}}
\pgfpatharc{205}{42}{1.170625\du and 1.170625\du}
\pgfusepath{stroke}
}
\pgfsetlinewidth{0.100000\du}
\pgfsetdash{{1.000000\du}{1.000000\du}}{0\du}
\pgfsetdash{{1.000000\du}{1.000000\du}}{0\du}
\pgfsetbuttcap
{
\definecolor{dialinecolor}{rgb}{0.000000, 0.000000, 0.000000}
\pgfsetfillcolor{dialinecolor}
% was here!!!
\definecolor{dialinecolor}{rgb}{0.000000, 0.000000, 0.000000}
\pgfsetstrokecolor{dialinecolor}
\draw (18.200000\du,2.962500\du)--(27.950000\du,2.912500\du);
}
\pgfsetlinewidth{0.100000\du}
\pgfsetdash{}{0pt}
\pgfsetdash{}{0pt}
\pgfsetbuttcap
{
\definecolor{dialinecolor}{rgb}{0.000000, 0.000000, 0.000000}
\pgfsetfillcolor{dialinecolor}
% was here!!!
\definecolor{dialinecolor}{rgb}{0.000000, 0.000000, 0.000000}
\pgfsetstrokecolor{dialinecolor}
\pgfpathmoveto{\pgfpoint{19.950016\du}{1.812506\du}}
\pgfpatharc{292}{159}{0.849480\du and 0.849480\du}
\pgfusepath{stroke}
}
% setfont left to latex
\definecolor{dialinecolor}{rgb}{0.000000, 0.000000, 0.000000}
\pgfsetstrokecolor{dialinecolor}
\node[anchor=west] at (0.500000\du,6.112500\du){P0=(0,0)};
% setfont left to latex
\definecolor{dialinecolor}{rgb}{0.000000, 0.000000, 0.000000}
\pgfsetstrokecolor{dialinecolor}
\node[anchor=west] at (5.650000\du,4.062500\du){A1};
% setfont left to latex
\definecolor{dialinecolor}{rgb}{0.000000, 0.000000, 0.000000}
\pgfsetstrokecolor{dialinecolor}
\node[anchor=west] at (8.900000\du,0.862500\du){A2};
% setfont left to latex
\definecolor{dialinecolor}{rgb}{0.000000, 0.000000, 0.000000}
\pgfsetstrokecolor{dialinecolor}
\node[anchor=west] at (16.350000\du,0.762500\du){A3};
% setfont left to latex
\definecolor{dialinecolor}{rgb}{0.000000, 0.000000, 0.000000}
\pgfsetstrokecolor{dialinecolor}
\node[anchor=west] at (14.800000\du,2.562500\du){};
% setfont left to latex
\definecolor{dialinecolor}{rgb}{0.000000, 0.000000, 0.000000}
\pgfsetstrokecolor{dialinecolor}
\node[anchor=west] at (18.150000\du,1.812500\du){};
% setfont left to latex
\definecolor{dialinecolor}{rgb}{0.000000, 0.000000, 0.000000}
\pgfsetstrokecolor{dialinecolor}
\node[anchor=west] at (17.850000\du,2.062500\du){A4};
% setfont left to latex
\definecolor{dialinecolor}{rgb}{0.000000, 0.000000, 0.000000}
\pgfsetstrokecolor{dialinecolor}
\node[anchor=west] at (4.850000\du,1.162500\du){L1};
% setfont left to latex
\definecolor{dialinecolor}{rgb}{0.000000, 0.000000, 0.000000}
\pgfsetstrokecolor{dialinecolor}
\node[anchor=west] at (12.600000\du,-2.287500\du){L2};
% setfont left to latex
\definecolor{dialinecolor}{rgb}{0.000000, 0.000000, 0.000000}
\pgfsetstrokecolor{dialinecolor}
\node[anchor=west] at (20.050000\du,0.362500\du){L3};
% setfont left to latex
\definecolor{dialinecolor}{rgb}{0.000000, 0.000000, 0.000000}
\pgfsetstrokecolor{dialinecolor}
\node[anchor=west] at (7.100000\du,-2.337500\du){P1=(W1,Z1)};
% setfont left to latex
\definecolor{dialinecolor}{rgb}{0.000000, 0.000000, 0.000000}
\pgfsetstrokecolor{dialinecolor}
\node[anchor=west] at (18.175000\du,-2.337500\du){P2=(W2,Z2)};
% setfont left to latex
\definecolor{dialinecolor}{rgb}{0.000000, 0.000000, 0.000000}
\pgfsetstrokecolor{dialinecolor}
\node[anchor=west] at (20.825000\du,3.912500\du){P3=(W,Z)};
% setfont left to latex
\definecolor{dialinecolor}{rgb}{0.000000, 0.000000, 0.000000}
\pgfsetstrokecolor{dialinecolor}
\node[anchor=west] at (20.925000\du,-2.637500\du){};
\pgfsetlinewidth{0.100000\du}
\pgfsetdash{}{0pt}
\pgfsetdash{}{0pt}
\pgfsetbuttcap
{
\definecolor{dialinecolor}{rgb}{0.000000, 0.000000, 0.000000}
\pgfsetfillcolor{dialinecolor}
% was here!!!
\definecolor{dialinecolor}{rgb}{0.000000, 0.000000, 0.000000}
\pgfsetstrokecolor{dialinecolor}
\draw (3.675000\du,6.212500\du)--(17.725000\du,-1.487500\du);
}
% setfont left to latex
\definecolor{dialinecolor}{rgb}{0.000000, 0.000000, 0.000000}
\pgfsetstrokecolor{dialinecolor}
\node[anchor=west] at (11.775000\du,2.762500\du){HL};
% setfont left to latex
\definecolor{dialinecolor}{rgb}{0.000000, 0.000000, 0.000000}
\pgfsetstrokecolor{dialinecolor}
\node[anchor=west] at (3.225000\du,-8.087500\du){z};
% setfont left to latex
\definecolor{dialinecolor}{rgb}{0.000000, 0.000000, 0.000000}
\pgfsetstrokecolor{dialinecolor}
\node[anchor=west] at (27.125000\du,6.062500\du){w};
\pgfsetlinewidth{0.100000\du}
\pgfsetdash{}{0pt}
\pgfsetdash{}{0pt}
\pgfsetbuttcap
{
\definecolor{dialinecolor}{rgb}{0.000000, 0.000000, 0.000000}
\pgfsetfillcolor{dialinecolor}
% was here!!!
\definecolor{dialinecolor}{rgb}{0.000000, 0.000000, 0.000000}
\pgfsetstrokecolor{dialinecolor}
\pgfpathmoveto{\pgfpoint{8.425000\du}{6.212503\du}}
\pgfpatharc{358}{321}{3.739603\du and 3.739603\du}
\pgfusepath{stroke}
}
% setfont left to latex
\definecolor{dialinecolor}{rgb}{0.000000, 0.000000, 0.000000}
\pgfsetstrokecolor{dialinecolor}
\node[anchor=west] at (8.825000\du,4.762500\du){HA};
\end{tikzpicture}
}
	\caption{Armens vinklar}
\end{figure}

Omvandlingen sker på följande sätt där gripperRotationOffset och gripper ignoreras. Antag att roboten startar i $(X,Y,Z,GR)=(0,0,0,\pi/2)$.
Där $GR$(GripRadian) är vilken vinkel gripklon har till $z$-axeln.\newline
Vi omvandlar våra rum-koordinater till en vinkel och plankoordinater $X,Y,Z,GR\rightarrow A,W,Z,GR$. Där $A$ är armens vinkel mot positiva $x$-axeln och $W$ är armens utsträckning.
\todo{Visualisera omvandling rumkoordinater till plan}
$$A=tan^{-1}(\dfrac{Y}{X}) $$
$$W=\sqrt{X^2+Y^2}$$
Genom denna omvandlingen så kan vi göra om detta till ett 2-dimensionellt problem istället för ett 3-dimensionellt problem. Vi får in $X$,$Y$,$Z$,$GR$ och ställer in robotens vinkel mot $x$-axelns positiva del och sedan ser vi till att roboten bara har rätt utsträckning och höjd.
$$P_4=(W,Z)$$
$$P_0=(0,0)$$
$$W_2=W-cos(GR)*L_3$$
$$Z_2=Z-sin(GR)*L_3$$
$$H_L=\sqrt{W_2^2+Z_2^2}$$
$$H_A=tan^{-1}(\dfrac{Z_2}{W_2})$$
Cosinussatsen ger:
$$A_1=cos^{-1}(\dfrac{L_1^2+H_L^2-L_2^2}{2*L_1*H_L})+H_A$$
$$W_1=cos(A_1)*L_1$$
$$Z_1=sin(A_1)*L_1$$
$$A_2=tan^{-1}(\dfrac{Z_2-Z_1}{W_2-W_1})-A_1$$
$$A_3=GR-D_2-D_3$$

Då är alla vinklar beräknade men servornas nollpunkter är inte på samma ställen som nollställena i beräkningarna ovanför. Så följande offset ställs in i grader:
\begin{itemize}
	\item $A0=A0+60$
	\item $A1=240-A1$
	\item $A2=240-A2$
	\item $A3=150-A3$
	\item $A4=240-A0+GripperRotationOffset$ (klons rotation beror på basens rotation)
	\item $A5$ har ingen offset
\end{itemize}
Dessa vinklar är nu i grader. Dessa multipliceras nu med $3.41$ \todo{Varför?} och avrundas nedåt till heltal och kan nu användas direkt av styrenheten.
