% Analys av systemet. Vad kunde ha gjorts bättre? Vad kan utvecklas vidare?

\section{Slutsatser}

Vi har byggt en robot som kan utföra en rad uppgifter både autonomt och manuellt. Även om systemet 
fungerar felfritt så skulle många förbättrningar kunna göras. 
Mjukvarumässigt kan en hel del kod optimeras för att systemet ska flyta på bättre, speciellt på avr:en 
där styrningen av servona sker. I nuvarande version av systemet kan rörelsen av armen upplevas som 
aningen ryckig, detta skulle kunna fixas med förbättrad mjukvara på ovan nämnda avr. På 
beagleboarden återfinnes också kod för detta som kan göras bättre. Filtreringsfunktioner för 
sensorvärdena kan även optimeras för att undvika felaktiga värden. All kod kan göras snyggare för 
utökad läsbarhet.

Ur ett hårdvarumässigt perspektiv skulle fler och mer precisa sensorer (mer specifikt 
avståndssensorerna) kunna förbättra systemet.

Om vi hade haft mer tid hade vi velat utöka funktionaliteten hos roboten. En autonom upplockning av 
paket hade vart önskvärd och hade gjort robotens autonoma läge helt komplett.

Som sagt fungerar systemet felfritt men med ovan nämnda förbättringar i mjukvaran samt hårdvaran 
kan produkten bli ännu mer attraktiv på marknaden.


\todo{Reflektion. Hur kan systemet förbättras?}
