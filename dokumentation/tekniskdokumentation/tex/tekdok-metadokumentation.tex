% Hur vi har dokumenterat projektet

\section{Metadokumentation}

All dokumentation utöver den i koden är skriven i \LaTeX. En dokumentmall hämtades ursprungligen från kurshemsidan för kursen TATA62\cite{lipsmall}. Den har sedan redigerats och vidareutvecklats för att passa den här kursen och våra syften.

All dokumentation följer LIPS-standarden angiven i boken Projektmodellen Lips\cite{LIPS}.

\subsection{Dokumentation av kod}

All kod skriven i C följer vår kodstandard \todo{bilaga kodstandard}, kod skriven i Python följer PEP 8 och i båda fallen är den väl kommenterad \todo{Kommentera all kod väl}. I det här dokumentet dokumenteras endast hur kodbasen fungerar och dess struktur översiktligt.

\subsection{Mappstruktur}

\begin{figure}[h!]
	\dirtree{%
	.1 /.
	.2 src.
	.3 main\_unit.
	.3 spi.
	.3 tools.
	.3 pc.
	.3 drive\_unit.
	.3 sensor\_unit.
	.2 dokumentation.
	.3 systemskiss.
	.3 designspecifikation.
	.3 projektplan.
	.3 tekniskdokumentation.
	.3 kopplingsschema.
	.3 mall.
	.3 tidsplan.
	.3 motesprotokoll.
	.3 statusrapport.
	}
	\caption{Vår mappstruktur. \todo{Övermeta?}}
	\label{fig:dir}
\end{figure}
