% Hur huvudenheten är designad

\section{Huvudenhet}

\todo{Vad fyller Huvudenheten för funktion?}

\subsection{Hårdvara}
Den hårdvara som ingår i huvudenheten består av en Beagleboard-xm. På denna sitter en Blåtands-enhet av typ \todo{Typ/modell av BT?} monterad för kommunikation med PCenheten. Denna är ansluten till styrenheten och sensorenheten med en flatkabel som innehåller SPI-busen.
\todo{Bild på BB?}

\subsection{Mjukvara}
Mjukvaran är uppdelad i fyra trådar. En maintråd som har hand om all styrlogik, en pctråd som hanterar all kommunikation med PC:n, en sensortråd som kontinuerligt uppdaterar huvudenhetens sensorvärden och en regulatortråd som har hand om regleringen så roboten kan följa banan utan problem. Alla dessa kommunicerar med varandra genom ett globalt dictionary som innehållar alla olika variabler som behöver delas. Kommandon från PC:n delas mellan pctråden och maintråden genom en kö.

\todo{Schema över trådar}

\subsubsection{Maintråd}
\todo{Hur tolkar vi sensordata?}
\todo{bäst om dennis skriver denna}

\subsubsection{PCtråd}
PCtråden sätter upp en socket när den skapas och väntar på en inkommande anslutning från en PC. När den får en anslutning så väntar den på ett kommando.

\todo{Följande står redan i protokoll} Kommandot kommer som en textsträng på formatet \textit{;kommando=argument1,argument2;}. Om kommandot inte har några argument så ser strängen ut på följande sätt \textit{:"kommando"}.

Denna sträng görs om till en lista med följande utseende \textit{: ["kommando",["argument1","argument2"]]}.

Oftast tas flera kommandon emot samtidigt och PCtråden gör om dessa till en lista av kommandon som den går igenom och behandlar ett efter ett. Får man till exempel in två kommandon om sätta motorhastigheten så ser tillvägagångssättet ut på följande sätt:
\begin{enumerate}
\item ";motorSpeed=speed1,speed2;motorspeed=speed3;speed4;" tas emot
\item detta görs om till [["motorspeed",["speed1","speed2"]],["motorSpeed",["speed3","speed4"]]]
\item det första kommandot behandlas genom att argumenten omvandlas till heltal, motorernas hastighet uppdateras i dictionaryt och kommandot läggs till i kön så maintråden vet att den ska skicka ut kommandot.
\item nästa kommando behandlas på samma sätt men om speed1=speed3 och speed2=speed4 så ignoreras kommandot då det inte påverkar hastigheterna. \todo{Om speed1 != speed3, skrivs speed1 över? Skickar maintråden speed3 2 ggr?}
\end{enumerate}

\subsubsection{Sensortråd}
Sensortråden hämtar data från Sensorenheten med en given uppdateringsfrekvens. De hämtade sensorvärdena lagras i den globala datastrukturen.

\subsubsection{Regulatortråd}
Regulatortråden använder data från linjesensorerna för att beräkna hur roboten behöver röra sig för att följa linjen. Detta görs sedan om till motsvarande motorhastigheter och lagras i den globala datastrukturen.

Regleringen utförs i en egen tråd för att säkerhetsställa att beräkningarna sker med jämna tidsintervall.
