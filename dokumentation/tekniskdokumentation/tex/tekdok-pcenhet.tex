% Hur är PC-enheten designad?

\section{PCenhet}
Syftet med PCenheten är att sköta kommunikationen mellan en PC och huvudenheten samt
underlätta för användaren att styra systemet genom ett smart och tydligt grafiskt gränssnitt.
Mjukvaran för att göra detta är uppdelad i två delar: En PC-modul som hanterar all data som
skickas och tas emot samt ett grafiskt gränssnitt (GUI) som användaren interagerar med. Allt är skrivet i
programspråket Python.

\subsection{Hårdvara}
För att köra PCenheten krävs en PC med Python installerat. Under utvecklingen av systemet har operativsystemet Ubuntu använts och testats. Vidare krävs en Blåtands-enhet för kommunikation med roboten.

\subsection{PC-modul}
PC-modulens uppgift är att skapa ett gränssnitt för kommunikation mellan en PC och det övriga
systemet. Rent kodmässigt är den strukturerad i form av en klass med olika metoder för att skicka
och ta emot data från användaren och huvudenheten. Metoder som börjar på \texttt{get} läser olika data från
huvudenheten (hastighet, armposition, debugdata) och metoder som börjar på \texttt{set} sätter och skickar
kommando från GUI till huvudenheten. Klassens initiering tar in en sträng som är IP-adressen till
huvudenheten. För att skicka data över nätverket används funktioner från det inbyggda paketet
\texttt{socket}. Ett annat paket kallat \texttt{select} används för I/O verifiering.

\subsection{GUI}
Det grafiska gränssnittet ska hjälpa användaren att enkelt sköta och styra systemet.
Det är implementerat via en egen klass som använder funktioner ur paketet \texttt{tkinter} för att rita
upp och hantera olika grafiska objekt som fönster och knappar. Funktionerna kallar på lämpliga
metoder i PC-modulen om kommando ska skickas eller data ska läsas och visas i fönstret. För att
hantera indata från joystick och styrning används funktioner ur paketet \texttt{pygame}.
