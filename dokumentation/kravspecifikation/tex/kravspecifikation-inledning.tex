\section{Inledning}
Vi har fått i uppgift av beställaren att bygga ett system som ska kunna flytta paket på ett lager. Systemet ska följa en bana enligt uppsatta banregler (se Bilaga \ref{banregler}) och flytta paket mellan uppsatta stationer.

\subsection{Parter}
Systemet har beställts av köparen, Tomas Svensson. Leverantör är Grupp 2.

\subsection{Syfte och mål}
Målet med projektet är att konstruera ett system som autonomt ska kunna röra sig i ett lager. Från en dator ska systemet kunna styras att plocka upp paket.

\subsection{Användning}
Systemet ska sättas vid en startposition, enligt regler definierade i banreglerna. När systemet sedan slås på följer roboten banan till nästa station där ett paket skall plockas upp eller sättas ner. Vid stationen styr användaren systemet från en dator trådlöst för att plocka upp paket. Nedsättning av paket sker autonomt.

\subsection{Bakgrundsinformation}
Vi är studenter vid Linköpings Universitet som läser kursen TSEA29. Vår examinator agerar beställare och har givit oss i uppdrag att konstruera en lagerrobot enligt givet projektdirektiv.

\subsection{Definitioner}

\begin{itemize}
\item{Vi har beslutat att kalla vårt system GLORIA}
\item{Prioritetsnivå 1: Krav som skall ingå i systemet}
\item{Prioritetsnivå 2: Krav som skall ska ingå i systemet om tid finns}
\end{itemize}
