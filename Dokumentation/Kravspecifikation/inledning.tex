\section{Inledning}
Vi har fått i uppgift av beställaren att bygga ett system som ska kunna användas på ett lager. Systemet ska följa en bana enligt uppsalla banregler (se Bilaga \ref{banregler}) och flytta paket mellan uppsatta stationer.

\begin{LIPSkravlista}
\LIPSkrav{Orginal}{Systemet ska kunna följa en bana enligt Bilaga \ref{banregler}}{1}
\end{LIPSkravlista}

\subsection{Parter}
Systemet har beställts av köparen, Tomas Svensson. Leverantör är Grupp 2.

\subsection{Syfte och mål}
Målet med projektet är att konstruera ett system som autonomt ska kunna röra sig i ett lager. Från en dator ska systemet kunna styras att plocka upp och sätta ner paket.

\subsection{Användning}
Systemet ska sättas vid en startposition, enligt regler definierade i banreglerna. När systemet sedan slås på följer roboten banan till nästa station. Vid stationen styr användaren systemet från en dator via blåtand och avgör om uppplockning eller nedsättning skall ske.

\subsection{Bakgrundsinformation}
Vi är studenter vid Linköpings Universitet som läser kursen TSEA29. Vår examinator agerar beställare och har givit oss i uppdrag att konstruera en lagerrobot enligt givet projektdirektiv

\subsection{Definitioner}

\begin{itemize}
\item{Vi har beslutat att kalla vårt system GLORIA.}
\item{Prioritetsnivå 1: Krav som skall ingå i systemet}
\item{Prioritetsnivå 2: Krav som vi vill ska ingå i systemet}
\end{itemize}
