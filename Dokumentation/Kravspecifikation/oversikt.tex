\section{Översikt av systemet}

\subsection{Grov beskrivning av produkten}
Systemet representerar en lagerrobot som ska kunna navigera autonomt med hjälp av en tejplinje och hitta paketstationer. Armen och motorerna ska kunna styras manuellt av en användare.

\subsection{Produktkomponenter}
Den färdiga produkten kommer innehålla följande komponenter
\begin{itemize}
\item{Robot}
\item{Programvara för robot}
\item{Programvara för att styra roboten från en dator}
\item{Teknisk dokumentation}
\item{Användarhandledning}
\end{itemize}

\subsection{Beroenden tilll andra system}
Gloria behöver en PC för att kunna fjärrstyras trådlöst.

\subsection{Ingående delsystem}
Systemet består av sex delsystem. En PC-modul som består av mjukvara för att styra roboten externt. En huvudmodul som styr fyra undermoduler: en kommunikationsmodul, en sensormodul, en armmodul och en motormodul. \\

\tikzstyle{block} = [draw, fill=blue!20, rectangle, 
    minimum height=3em, minimum width=6em]
\tikzstyle{sum} = [draw, fill=blue!20, circle, node distance=1cm]
\tikzstyle{input} = [coordinate]
\tikzstyle{output} = [coordinate]
\tikzstyle{pinstyle} = [pin edge={to-,thin,black}]

% The block diagram code is probably more verbose than necessary
\begin{center}
\begin{tikzpicture}[auto, node distance=2cm,>=latex']
    % We start by placing the blocks
    \node [block] (PC) {PC};
    \node [block, below of=PC, node distance=2cm] (Kommunikationsmodul) {Kommunikationsmodul};
    \node [block, below of=Kommunikationsmodul, node distance=2cm] (Huvudmodul) {Huvudmodul};
    \node [block, right of=Huvudmodul, node distance=3cm] (Motormodul) {Motormodul};
    \node [block, below of=Huvudmodul, node distance=2cm] (Armmodul) {Armmodul};
    \node [block, left of=Huvudmodul, node distance=3cm] (Sensormodul) {Sensormodul};
    \draw [label=right:Blåtand] (PC) -- (Kommunikationsmodul);
    \draw (Kommunikationsmodul) -- (Huvudmodul);
    \draw (Huvudmodul) -- (Sensormodul);
    \draw (Huvudmodul) -- (Motormodul);
    \draw (Huvudmodul) -- (Armmodul);
\end{tikzpicture}
\end{center}

\subsection{Avgränsningar}
Roboten skall endast kunna köras på banor som följer banreglerna.

\subsection{Designfilosofi}
Funktionaliteten och driftsäkerheten av systemet prioriteras högst, dvs kunna leverera paketet till rätt plats utan problem.

\subsection{Generella krav på hela systemet}

\begin{LIPSkravlista}
\LIPSkrav{Orginal}{Roboten skall kunna färdas autonomt längs en bana}{1}
\LIPSkrav{Orginal}{Roboten skall stanna vid utmarkerade stationer}{1}
\LIPSkrav{Orginal}{Roboten skall, styrd av en användare, kunna plocka upp och sätta ner paket}{1}
\LIPSkrav{Orginal}{Roboten skall kunna ta emot kommandon trådlöst från en användare}{1}
\LIPSkrav{Orginal}{Roboten skall skicka sensor- och debugdata trådlöst till användaren}{1}
\LIPSkrav{Orginal}{Det skall finnas programvara för att skicka och ta emot data från roboten}{1}
\LIPSkrav{Orginal}{Det skall finnas möjlighet att ställa om roboten i ett läge där den detekterar, plockar upp och lägger ner paket autonomt}{2}
\end{LIPSkravlista}
