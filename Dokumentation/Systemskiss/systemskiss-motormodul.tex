\section{Motorenhet}

Motorenheten har till uppgift att driva de motorer som driver hjulen och de servon som styr armen.

\subsection{Framdrivning}

Motorenheten innehåller två hjulpar, de styrs separat för att göra det möjligt att styra. Huvudmodulen skickar kommandon för hur snabbt vardera hjulpar skall köras, motorenheten ser sedan till att motorerna kör i den efterfrågade hastigheten.

\subsubsection{Arbetsblock}

\todo{Lista över saker motormodulen skall kunna göra, relaterat till framdrivning}

\subsection{Robotarm}

Robotarmen består av 7 servon av modell AX12-A. Dessa styrs genom att en målvinkel sätts (0-1023), med möjlighet att ändra hastighet, vridmoment och styra av/på. Från huvudmodulen får enheten målvinklar för varje enskild led. Motormodulen är ansvarig för att se till att parallella servon körs synkroniserat, för att inte slita sönder varandra. \todo{Motormodulen är också ansvarig för att se till att armen ej kolliderar med roboten.} 

\todo{Karta över arm med numrerade servon}

\subsubsection{Arbetsblock}

\todo{Lista över saker motormodulen skall kunna göra, relaterat till robotarm}

\subsection{Processor}

Eftersom motorenheten kommunicerar med huvudmodulen genom UART kommer kommandon att tolkas med avbrott.